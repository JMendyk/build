\section{Build Systems \`a la Carte}\label{sec-build}

The focus of this paper is on a variety of implementations of
\hs{Build}~\hs{c}~\hs{i}~\hs{k}~\hs{v}, given
a \emph{client-supplied} implementation of \hs{Tasks}~\hs{c}~\hs{k}~\hs{v}. That
is, we are going to take \hs{Tasks} as given from now on, and explore variants of
\hs{Build}: first abstractly (in this section) and then concretely
in~\S\ref{sec-implementations}.

As per the definition of minimality~\ref{def-minimal}, a minimal build
system must \textbf{rebuild only out-of-date keys} and at most once. The only
way to achieve the ``at most once'' requirement while producing a correct build
result (\S\ref{sec-build-correctness}) is to \textbf{build all keys in an
order that respects their dependencies}.

\vspace{1mm}
We have emboldened two different aspects above: the part of the
build system responsible for scheduling tasks in the dependency order
(a `scheduler') can be cleanly separated from the part responsible for deciding
whether a key needs to be rebuilt (a `rebuilder'). We tackle each
aspect separately in subsections~\S\ref{sec-dependency-orderings}
and~\S\ref{sec-out-of-date}.

\subsection{The scheduler: respecting the dependency order}
\label{sec-dependency-orderings}

Section \S\ref{sec-background} introduced three different \emph{task schedulers}
that decide which tasks to execute and in what order; see the ``Scheduler'' column
of Table~\ref{tab-summary} in \S\ref{sec-background-summary}.
This subsection explores the properties of the three schedulers, and
possible implementations.

\vspace{-2mm}
\subsubsection{Topological}\label{sec-topological}

The topological scheduler pre-computes a linear order of tasks, which when followed,
ensures the build result is correct regardless of the initial store. Given a
task description and the output \hs{key}, you can compute the linear order by
first finding the (acyclic) graph of the \hs{key}'s reachable dependencies, and
then computing a topological sort. However, as we have seen in~\S\ref{sec-deps},
we can only extract dependencies from an applicative task, which requires the
build system to choose \hs{c}~\hs{=}~\hs{Applicative}, ruling out dynamic
dependencies.

\vspace{-2mm}
\subsubsection{Restarting}\label{sec-restarting}

To handle dynamic dependencies we can use the following approach: build tasks in
an arbitrary initial order, discovering their dependencies on the fly; whenever
a task calls \hs{fetch} on an out-of-date key \hs{dep}, abort the task, and
switch to building the dependency \hs{dep}; eventually the previously aborted
task is restarted and makes further progress thanks to \hs{dep} now being up to
date. This approach requires a way to abort tasks that have failed due to
out-of-date dependencies. It is also not minimal in the sense that a task may
start, do some meaningful work, and then abort.

To reduce the number of aborts (often to zero) \Excel records the
discovered task order in its \emph{calc chain}, and uses it as the
starting point for the next build (\S\ref{sec-background-excel}).
\Bazel's restarting scheduler does not store the discovered order
between build runs; instead, it stores the most recent task dependency
information. Since this information may become outdated, \Bazel may
also need to abort a task if a newly discovered dependency is out of date.

\vspace{-2mm}
\subsubsection{Suspending}\label{sec-suspending}

An alternative approach, utilised by the \hs{busy} build system
(\S\ref{sec-general-build}) and \Shake, is to simply build dependencies when
they are requested, suspending the currently running task. By combining that
with tracking the keys that have already been built, one can obtain a minimal
build system with dynamic dependencies.

This approach requires that a task may be started and then suspended until
another task is complete. Suspending can be done with cheap green threads and blocking
(the original approach of \Shake) or using continuation-passing style \citep{claessen_continuations} (what
\Shake currently does).

\subsection{The rebuilder: determining out-of-date keys} \label{sec-out-of-date}

Suppose the scheduler decides that a key should be brought up to date. The next
question is: does any work need to be done, or is the key already up to date?
Or, in a cloud-build system, do we have a cached copy of the value we need?
These questions can be addressed in one of four fundamental ways, with a number
of tweaks and variations within them.

\vspace{-2mm}
\subsubsection{A dirty bit}\label{sec-dirty-bit}

The idea of a dirty bit is to have one piece of persistent information per key,
saying whether the key is \emph{dirty} or \emph{clean}. After a build, all bits
are set to clean. When the next build starts, anything that changed between the
two builds is marked dirty. If a key and all its transitive dependencies are
clean, the key does not need to be rebuilt.

\Excel models the dirty bit approach most directly, having an actual dirty bit
associated with each cell, marking the cell dirty if the user modifies it.
It also marks dirty all cells that (transitively) depend on the modified cell.
\Excel does not record dynamic dependencies of each cell; instead it computes a
\emph{static over-approximation} -- it is safe for it to make more cells dirty
than necessary, but not vice versa. The over-approximation is as follows: a cell
is marked dirty if its formula statically refers to a dirty cell, or if the
formula calls a function like \cmd{INDIRECT} whose dependencies cannot be
guessed from the formula alone. The over-approximation is clear for
\cmd{INDIRECT}, but it is also present for \cmd{IF}, where both branches are
followed even though dynamically only one is used.

\Make uses file modification times, and compares files to their dependencies,
which can be thought of as a dirty bit which is set when a file is older than
its dependencies. The interesting property of this dirty bit is that it is not
under the control of \Make; rather it is existing file-system information that
has been repurposed. Modifying a file automatically clears its dirty bit, and
automatically sets the dirty bit of the files depending on it (but not
recursively). Note that \Make requires that file timestamps only go forward in
time, which can be violated by backup software.

With a dirty bit it is possible to achieve minimality. However, to achieve early
cutoff (\S\ref{sec-background-shake}) it would be important to reset the dirty
bit after a computation that did not change the value and make sure that cells
that depend on it are not rebuilt unnecessarily. For \Excel, this is difficult
because the dependent cells have already been recursively marked dirty. For
\Make, it is impossible to mark a file clean and at the same time not mark the
keys that depend on it dirty. \Make can approximate early cutoff by not
modifying the result file, and not marking it clean, but then it will be rebuilt
in every subsequent build.

\vspace{-2mm}
\subsubsection{Verifying traces}\label{sec-verifying-traces}

An alternative way to determine if a key is dirty is to record the
values/hashes of dependencies used last time, and if something has changed, the
key is dirty and must be rebuilt~--~in essence a \emph{trace} which we can use
to \emph{verify} existing values. For traces, there are two essential
operations~--~adding a new value to the trace store, and using the traces to
determine if a key needs rebuilding. Assuming a store of verifying traces
\hs{VT}~\hs{k}~\hs{v}, the operations are:

\begin{minted}[fontsize=\small,xleftmargin=10pt]{haskell}
recordVT@\,@::@\,@k -> Hash v -> [(k, Hash v)] -> VT k v -> VT k v
verifyVT@\,@::@\,@(Monad m,@\,@Eq k,@\,@Eq v) => k -> Hash v -> (k -> m (Hash v)) -> VT k v -> m Bool
\end{minted}

\noindent
Rather than storing (large) values \hs{v}, the verifying trace \hs{VT} stores
only hashes, of type \hs{Hash}~\hs{v}, of those values. Since the verifying
trace persists from one build to the next -- it constitutes the build system's
``memory'' -- it is helpful for it to be of modest size. After successfully
building a key, we call \hs{recordVT} to add a record to the current \hs{VT},
passing the key, the hash of its value, and the list of hashes and dependencies.

More interestingly, to \emph{verify} whether a key needs rebuilding we supply
the key, the hash of its current value, a function for obtaining the post-build
value of any key (using a scheduling strategy as per
\S\ref{sec-dependency-orderings}), and the existing \hs{VT} information. The
result will be a \hs{Bool} where \hs{True} indicates that the current value is
already up to date, and \hs{False} indicates that it should be rebuilt.

One potential implementation would be to record all arguments passed to
\hs{recordVT} in a list, and verify by simply checking if any list item matches
the information passed by \hs{verifyVT}.  We discuss smarter implementations
in~\S\ref{sec-smart-traces}.

A verifying trace, and other types of traces discussed in this section, support
dynamic dependencies and minimality; furthermore, all traces except for deep
traces~(\S\ref{sec-deep-constructive-traces}) support early cutoff.

\subsubsection{Constructive traces}\label{sec-constructive-traces}

A verifying trace deliberately records only small hashes, so that it can be small.
A \emph{constructive} trace additionally stores the resulting value.
Once we are storing the complete result it makes sense
to record many constructive traces per key, and to share them with other users,
providing cloud-build functionality. We can represent this additional
information by providing the operations:

\begin{minted}[fontsize=\small,xleftmargin=10pt]{haskell}
recordCT    :: k -> v -> [(k, Hash v)] -> CT k v -> CT k v
constructCT :: (Monad m, Eq k, Eq v) => k -> (k -> m (Hash v)) -> CT k v -> m [v]
\end{minted}

\noindent
The function \hs{recordCT} looks like \hs{recordVT}, but instead of just passing
the hash of the resulting value, we require the actual value. The \hs{verifyVT}
has been replaced with \hs{constructCT}, which instead of taking the hash of the
current value as \emph{input}, returns a list of suitable values as \emph{output}.
If the current value in the store matches one of the possible values, the build
can skip this key. If the resulting list is empty, the key must be rebuilt.
However, if the current value does not match the store, and there is a possible
value, we can use any value from the constructive list \emph{without} doing any
work to build it, and copy it into the store.

\subsubsection{Deep constructive traces}\label{sec-deep-constructive-traces}

Constructive traces always verify keys by looking at their immediate
dependencies, which must have first been brought up to date, meaning that the
time to verify a key depends on the number of transitive dependencies. A
\emph{deep} constructive trace optimises this process by only looking at the
terminal \emph{input keys}, ignoring any intermediate dependencies. The operations
capturing this approach are the same as for constructive traces
in~\S\ref{sec-constructive-traces}, but we use the names \hs{recordDCT} and
\hs{constructDCT}, where the underlying \hs{DCT} representation need only record
information about hashes of inputs, not intermediate dependencies.

Current build systems using deep constructive traces always record hashes of
terminal \emph{input keys}, but the technique works equally well if we skip any
number of dependency levels (say $n$ levels). The input-only approach is the
special case of $n = \infty$, and constructive traces are the special case of
$n = 1$. When $n \ge 2$, deep constructive traces require the tasks to be
\emph{deterministic}, as otherwise it is possible to violate correctness, as
illustrated by an example in~\S\ref{sec-cloud-aspects}.

A downside of deep constructive traces is that they cannot support early cutoff
(\S\ref{sec-background-shake}), other than at $n$ levels of dependencies. On the
other hand, these traces are particularly useful for \emph{shallow builds}, as
discussed in~\S\ref{sec-cloud-aspects}.

\subsection{Smarter \hs{[Trace]} data structures}\label{sec-smart-traces}

In the examples above, we have used abstract types for the traces. Concretely, in our example code, they are all recorded as lists of:

\begin{minted}[xleftmargin=10pt]{haskell}
data Trace k v r = Trace { key :: k, depends :: [(k, Hash v)], result :: r }
\end{minted}

Here \hs{r} is either \hs{Hash}~\hs{v} (verifying traces) or
\hs{v} (constructive traces). A real system is highly likely to use a more
optimised implementation. Some of the most obvious optimisations are:

\begin{itemize}
\item Any system using verifying traces is unlikely to see significant benefit
from storing more than one \hs{Trace} per key\footnote{There is a small chance
of a benefit if the dependencies change but the result does not, and then the
dependencies change back to what they were before.}. Therefore, such systems can
store \hs{Map}~\hs{k}~\hs{(Trace}~\hs{k}~\hs{v}~\hs{(Hash}~\hs{v))}, where the
initial \hs{k} is the \hs{key} field of \hs{Trace}.

\item Any system using \hs{Applicative} dependencies can omit the dependency
keys from the \hs{Trace} since they can be recovered from the \hs{key} field.

\item Any \hs{Applicative} build system using constructive traces, e.g.
\CloudBuild, can index directly from the key and results to the output result~--~i.e.
\hs{Map}~\hs{(@@k,}~\hs{[Hash}~\hs{v])}~\hs{v}. Importantly, assuming the traces
are stored on a central server, the client can compute the key and the hashes of
its dependencies, then make a single call to the server to retrieve the result.

\item Many cloud build systems store hashes of values in the trace information,
then have a separate content-addressable cache which associates hashes with
their actual contents.
\vspace{-1mm}
\end{itemize}

\subsection{Build Systems \`a la Carte}\label{sec-design-space}

\begin{table}
\vspace{-1mm}
\smaller
\centering
\begin{tabular}{l||c|c|c}
\hline
 & \multicolumn{3}{c}{Scheduling algorithm} \\
Rebuilding strategy  & Topological\hspace{2mm}\S\ref{sec-topological} & Restarting\hspace{2mm}\S\ref{sec-restarting} & Suspending\hspace{2mm}\S\ref{sec-suspending}    \\\hline
\hline
Dirty bit\hfill\S\ref{sec-dirty-bit}                                                             & \Make       & \Excel & -              \\\hline
Verifying traces\hfill\S\ref{sec-verifying-traces}                                               & \Ninja      & -      & \Shake         \\\hline
Constructive traces\hspace{2mm}\hfill\S\ref{sec-constructive-traces}                             & \CloudBuild & \Bazel & -              \\\hline
Deep constructive traces\hspace{2mm}\hfill\S\ref{sec-deep-constructive-traces} & \Buck       & -      & \Nix           \\\hline
\end{tabular}
\vspace{1mm}
\caption{Build systems \`a la carte.\label{tab-build-systems}}
\vspace{-8mm}
\end{table}

With the information in this section we can build Table~\ref{tab-build-systems},
which tabulates combinations of the scheduling algorithm and the rebuilding
strategy, providing 12~possible build systems, 8~of which are inhabited
by existing build systems (we discuss these systems in~\S\ref{sec-background} and
\S\ref{sec-related-build}). Of the remaining 4~spots, all result in workable
build systems. The most interesting unfilled spot in the table is suspending
constructive traces, which would provide many benefits, and which we title
\Cloud \Shake and explore further in~\S\ref{sec-implementation-cloud}.
% (as we plan on extending \Shake to occupy that spot)
