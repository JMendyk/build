\section{Introduction}\label{sec-intro}

Build systems (such as \Make) are big, complicated, and used by every
software developer on the planet.  But they are a sadly unloved part
of the software ecosystem, very much a means to an end, and seldom the
focus of attention.
Seldom do people ask questions like ``What does it mean for my build
system to be correct?'' or ``What are the trade-offs between different
approaches?''.  Moreover, complex build systems use subtle algorithms, but they
are often hidden away, and not the object of study.

Recently, the challenges of scale have driven large software firms
like Microsoft, Facebook, and Google to develop their own build
systems, each with its own choices and idiosyncrasies.
In this paper we offer a general framework in which to understand
and compare build systems,
in a way that is both abstract (omitting incidental detail)
and yet precise (implemented as Haskell code).  Specifically we make
these contributions:
\begin{itemize}
\item Build systems vary on many axes,
including: static vs dynamic dependencies; cloud-build, including
shallow vs deep; deterministic vs non-deterministic build rules;
early cut-off; self-tracking build systems; and persistent metadata.
In \S\ref{sec-background} we identify some key properties, in the context of
four carefully-chosen build systems.
\item We describe some simple but novel abstractions that
  crisply encapsulate what a build system is (\S\ref{sec-abstractions}),
  allowing us, for example, to speak about what it means for a build system to be correct.

\item We identify two key design choices
  that are typically deeply built into any build system:
  \emph{the order in which dependencies are built} (\S\ref{sec-dependency-orderings})
  and \emph{whether or not a dependency is (re-)built} (\S\ref{sec-out-of-date}).
  These choices turn out to be orthogonal, which leads us to a new
  classification of the design space (\S\ref{sec-design-space}).

\item We show that we can instantiate
our abstractions to describe the essence of a variety of different
real-life build systems, including \Make, \Shake, \Bazel, and \Excel, each in
a dozen lines of code or so (\S\ref{sec-implementations}).
Doing this modelling in a single setting allows
the differences and similarities between these huge systems to be
brought out clearly.

\item Moreover, we can readily remix the ingredients to produce the first
  build system that supports both \emph{dynamic dependencies}
  and \emph{cloud build} (\S\ref{sec-implementation-cloud-shake}).

\end{itemize}
In short, instead of seeing build systems as unrelated
points in space, we now see them as locations in a landscape,
leading to better understanding of what they do and how they compare,
and suggesting exploration of other (as yet unoccupied points) in the
landscape.
We discuss engineering aspects in \S\ref{sec-engineering}, and related
work in \S\ref{sec-related}.

Papers about ``frameworks'' are often fuzzy.  This one is not: all our
abstractions are defined in Haskell, and we have (freely-available)
executable models of all the build systems we describe.