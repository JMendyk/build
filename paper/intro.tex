\section{Introduction}\label{sec-intro}

Build systems automate the execution of simple repeatable tasks for individual
users, as well as for large organisations. There are software build systems such
as \Make, \Ninja, \Shake, \Bazel, \Buck and many others, as well as various
incremental calculation engines, such as \Excel and \Calc. Existing build
systems vary in several different ways:

\begin{itemize}
    \item Many build systems, such as the venerable \Make, need to know the
    complete \emph{dependency graph} between tasks at the start of the build
    process. This makes it possible to analyse the graph ahead of time, which
    helps with efficient task scheduling but is fundamentally limited: parts of
    the dependency graph can often be discovered only during the build process,
    i.e. the dependency graph is \emph{dynamic}, not \emph{static}.

    \item When build systems are used by large teams, different team members
    often end up executing exactly the same tasks on their local machines.
    A \emph{cloud build system} can speed up builds dramatically by
    transparently sharing build results among team members. Furthermore, cloud
    build systems allow one to perform \emph{shallow builds} that materialise
    only end build products on a local machine, leaving all intermediates in the
    cloud. This is a significant optimisation compared to \emph{deep builds}
    that require all transitive dependencies of an end build product to be
    locally available. Non-cloud build systems cannot support shallow builds.

    \item Some build systems, e.g. \Buck, require all tasks to be
    \emph{deterministic}, i.e. produce exactly the same output when run on the
    same inputs. However, not all tasks are deterministic, and there are build
    systems that support \emph{non-determinism}. A simple example is \Excel's
    \textsf{RANDBETWEEN(low, high)} that returns a random integer in the
    interval \textsf{[low, high]}.

    \item If a build system executes a task and the result is unchanged from the
    previous build, it is unnecessary to execute the dependent tasks. We call
    this optimisation \emph{early cut-off}. Not all build systems support early
    cut-off: \Make and \Excel do not, whereas \Shake and \Buck do.

    \item Most build systems track changes of inputs and intermediate results,
    executing dependent tasks whenever they change, but some build systems can
    also track changes in the tasks themselves: if a task has changed, the build
    system will execute it. For example, when a cell's formula has changed,
    \Excel will recompute its value and propagate the changes. We call this
    \emph{self-tracking}. Self-tracking is uncommon in software build systems,
    where one often needs to manually initiate a rebuild of all tasks even if
    just a single task has changed.

    \item Most build systems persistently store auxiliary \emph{build
    information} for profiling and optimisation purposes: \Make stores file
    modification times, \Shake stores the discovered dynamic dependency graph,
    \Bazel and other cloud build systems store information about inputs and
    outputs of previously executed tasks, etc.
\end{itemize}

This paper presents a purely functional abstraction for build systems that
allows us to express all the above intricacies of build systems and design
complex build systems from simple primitives. The presented abstraction fits in
just two lines of Haskell code, which are explained
in~\S\ref{sec-build-abstractions}:

\begin{minted}{haskell}
type Compute c k v = @\std{forall}@ f. c f => (k -> f v) -> k -> f (Maybe v)
type Build c i k v = Compute c k v -> [k] -> Maybe i -> Map k v -> (i, Map k v)
\end{minted}

% It is known that build systems can take up to 27\% of software development
% effort, and that improvements to build systems rapidly pay off~\cite{build_maintenance}.
% Despite its importance, this subject is severely under-researched, which prompts
% major companies, such as Microsoft, Facebook and Google, to invest significant
% internal resources to make their own bespoke build system frameworks.

% Some build systems do not look like build systems but they are. A good example
% is spreadsheets, where cells play the role of files, and formulas play the role
% of build rules.

% We separate build systems into \emph{compute} and \emph{build} components,
% see~\S\ref{sec-build-abstractions}.
