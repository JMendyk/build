\section{Introduction}\label{sec-intro}

Build systems (such as make) are big, complicated, but unloved part of
the software ecosystem.  Every developer on the planet uses one, but
they are very much a means to an end, and seldom the focus of
attention.  But the challenges of scale have driven large software firms
like Microsoft, Facebook, and Google to develop their own build
systems, each with its own choices and idiosyncrasies.

Seldom do people ask questions like "What does it mean for my build
system to be correct?" or "What are the trade-offs between different
approaches?".  Complex build systems use subtle algorithms, but they
are often hidden away, and not the object of study.

In this paper we offer a general framework in which to discuss these
questions in a way that is both abstract (omitting incidental detail)
and yet precise (implemented as Haskell code).  Specifically we make
these contributions:
\begin{itemize}
\item We describe some simple but novel abstractions that
crisply encapsulate what a build system is.
\item We show that we can instantiate
these abstractions to describe the essence of a variety of different
build systems, including \Make, \Shake, \Bazel, and \Excel, each in
a dozen lines of code or so.
Doing this in a single setting allows
the differences and similarities between these huge systems to be
brought out clearly.
\item Build systems vary on many axes;
for example: static vs dynamic dependencies; cloud-build, including
shallow vs deep; deterministic vs non-deterministic build rules;
early cut-off; self-tracking build systems; and persistent metadata.
These properties (which we define in Section~\ref{sec-background}) are often
deeply-built-in assumptions of a particular build system.
In contrast, our framework allows them to be distinguished,
reasoned about, and varied.
\item Two particularly desirable properties are dynamic dependencies
and cloud-build; yet no currently-available build system supports
both.  Our framework makes it rather easy to do so, for the first
time.
\end{itemize}
Thus equipped, instead of seeing build systems as unrelated
points in space, we can re-envisage them as locations in a landscape,
leading to better understanding of what they do and how they compare,
and suggesting exploration of other (as yet unoccupied points) in the
landscape.

Papers about "frameworks" are often fuzzy.  This one is not: all our
abstractions are defined in Haskell, and we have (freely-available)
executable models of all the build systems we describe.  An unusual
feature is that we include Excel in our line-up because, looked at
in the right way, it certainly is a build system.

\todo{AM/NM}{Add a paragraph on the decomposition of build systems into
\emph{compute} and \emph{build} components, with the latter split into
\emph{when} and \emph{how} parts.}
