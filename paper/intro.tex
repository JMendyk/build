\section{Introduction}\label{sec-intro}

A build system is a critical component of most software projects, responsible
for compiling the source code written in various programming languages and
producing executable programs -- end software products. Build systems sound
simple, but they are not; in large software projects the build system grows from
simple beginnings into a huge, complex engineering artefact. Why? Because it
evolves every day as new features are continuously being added by software
developers; because it builds programs for a huge variety of target hardware
configurations (from mobile to cloud); and because it operates under strict
correctness and performance requirements, standing on the critical path between
the development of a new feature and its deployment into production.

It is known that build systems can take up to 27\% of software development
effort, and that improvements to build systems rapidly pay off~\cite{build_maintenance}.
Despite its importance, this subject is severely under-researched, which prompts
major companies, such as Microsoft, Facebook and Google, to invest significant
internal resources to make their own bespoke build system frameworks.

...

Some build systems do not look like build systems but they are. A good example
is spreadsheets, where cells play the role of files, and formulas play the role
of build rules.

...

We separate build systems into \emph{compute} and \emph{build} components,
see~\S\ref{sec-build-abstractions}.

...