\clearpage
\section{Engineering aspects}\label{sec-engineering}

Or ``Build systems in the real world'', or ``Crazy things section''...

\subsection{Non-deterministic computations}\label{sec-non-determinism}

Sandboxing: guarding against non-determinism due to missing dependencies.

\todo{AM}{Rewrite everything below.}

Non-determinism, specifically, its \emph{useless} form. This doesn't
include \Excel's \textsf{RANDBETWEEN} function, which is useful.
To define correctness of non-deterministic build systems, one needs to
switch from \hs{Applicative} and \hs{Monad} abstractions to
\hs{Alternative} and \hs{MonadPlus}, respectively, but this is not
required for actual implementations, most of which can happily accept
non-deterministic results (with a notable exception of \Buck).

... Some old text:

Build systems and spreadsheets compute output values from input and intermediate
values. In the most typical case, these \emph{computations} are \emph{functions},
such as \textsf{C1 = A1 + B1}, i.e. their result is uniquely determined by the
input values. However, in general they can be \emph{relations}, i.e. have
multiple valid results. A spreadsheet example: \textsf{A2 = A1 + RANDOM(1,6)}.
This computation has six valid results for each input value \textsf{A1}. In
build systems, the object file \textsf{obj/file.o} is sometimes not uniquely
determined by the source \textsf{src/file.c} -- different compiler runs may
produce different valid results.

Some build systems, e.g. \Buck, require all tasks to be
\emph{deterministic}, i.e. produce exactly the same output when run on the
same inputs. However, not all tasks are deterministic, and there are build
systems that support \emph{non-determinism}. A simple example is \Excel's
\textsf{RANDBETWEEN(low, high)} that returns a random integer in the
interval \textsf{[low, high]}.

\subsection{Concurrency}\label{sec-concurrency}

Nothing fundamentally difficult and fits our theory, but hard to get right in
practice.

\todo{NM}{Add a paragraph on how to introduce concurrency to implementations
in~\S\ref{sec-examples}}.

\subsection{Cloud aspects}\label{sec-cloud-aspects}

Cloud aspects: eviction/garbage collection, `frankenbuilds', etc.

\todo{AM}{Descibe frankenbuilds linking to CloudBuild paper.}

\subsection{Tracking and self-tracking}\label{sec-tracking-aspects}

Annotating dependency graphs with hashes, times, etc. is in many cases just an
engineering trade-off.

\todo{NM}{Clarify, e.g. Shake's \cmd{ChangeModtime}.}

\todo{NM}{Explain how self-tracking can be implemented.}

\subsection{Iterative computations}\label{sec-iterative-compute}

Iterative computations, e.g. LaTeX. \Excel has built-in support for this.

\todo{NM}{Clarify how this fits with the example implementations.}

\subsection{Polymorphism}\label{sec-polymorphism}

Polymorphic keys and values: tasks with multiple outputs, e.g. GHC's \cmd{hi}
and \cmd{o} files, phony tasks, etc.

\todo{AM}{Give an example.}

\todo{NM}{Describe how to add this feature to \Shake.}
