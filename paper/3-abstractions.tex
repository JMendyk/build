\section{Build systems, abstractly}\label{sec-abstractions}

This section presents purely functional abstractions that allow us to express
all the intricacies of build systems discussed in the previous
section~\S\ref{sec-background} and design complex build systems from simple
primitives. Specifically, we present the \emph{task} and \emph{build}
abstractions in~\S\ref{sec-task} and~\S\ref{sec-general-build}, respectively.
Sections~\S\ref{sec-build} and~\S\ref{sec-implementations} scrutinise the
abstractions further and provide concrete implementations for several build
systems.

\subsection{Common vocabulary for build systems}\label{sec-vocabulary}
\begin{figure}
\begin{minted}[fontsize=\small]{haskell}
-- An abstract store
data Store i k v
getInfo    :: Store i k v -> i
putInfo    :: Store i k v -> i -> Store i k v
getValue   :: Store i k v -> k -> v
putValue   :: Eq k => Store i k v -> k -> v -> Store i k v
getHash    :: Hashable v => Store i k v -> k -> Hash v
initialise :: i -> (k -> v) -> Store i k v
\end{minted}
\vspace{1mm}
\begin{minted}[fontsize=\small]{haskell}
-- Hashing
hash :: Hashable a => a -> Hash a
\end{minted}
\vspace{1mm}
\begin{minted}[fontsize=\small]{haskell}
-- Applicative functors
pure  :: Applicative f => a -> f a
(<$>) :: Applicative f =>   (a -> b) -> f a -> f b -- Left-associative
(<*>) :: Applicative f => f (a -> b) -> f a -> f b -- Left-associative
\end{minted}
\vspace{1mm}
\begin{minted}[fontsize=\small]{haskell}
-- Standard State monad from Control.Monad.State
data State s a
instance Applicative (State s)
instance Monad       (State s)
get       :: State s s
put       :: s -> State s ()
modify    :: (s -> s) -> State s ()
execState :: State s a -> s -> s
\end{minted}
\vspace{1mm}
\begin{minted}[fontsize=\small]{haskell}
-- Standard types from Data.Functor.Identity and Data.Functor.Const
newtype Identity a = Identity { runIdentity :: a }
newtype Const m a  = Const    { getConst    :: m }
\end{minted}
\vspace{0.5mm}
\begin{minted}[fontsize=\small]{haskell}
instance Functor (Const m) where
    fmap _ (Const m) = Const m
\end{minted}
\vspace{0.5mm}
\begin{minted}[fontsize=\small]{haskell}
instance Monoid m => Applicative (Const m) where
    pure _              = Const mempty
    Const x <*> Const y = Const (x <> y)
\end{minted}
\vspace{-3mm}
\caption{Signatures of main data types and library functions.}\label{fig-types}
\vspace{-4mm}
\end{figure}
We begin by establishing a common vocabulary for build systems.
% that allows us to abstract away from specific application domains, such as
% software development or spreadsheets.
Fig.~\ref{fig-types} provides a Haskell encoding of our main definitions,
and gives types of the library functions we use in this paper.

\vspace{1mm}
A \emph{build system} takes a \emph{task description}, a target \emph{key},
and a \emph{store}, and returns a new store in which the target key is
up to date. The store is a key/value map, together with \emph{persistent build
information} that is maintained by the build system itself.

\vspace{-2mm}
\paragraph{Keys and values.} In software build systems, keys are typically filenames,
e.g.~\cmd{main.c}, and values are file contents (a C program source
code in this case). In spreadsheets, keys are cell names,
e.g. \cmd{A1}, and values are numbers, text, etc. that are typically
displayed inside cells. In our Haskell code, we will use type
variables \hs{k} and \hs{v} to denote keys and values, respectively.

\vspace{-2mm}
\paragraph{Input, output, and intermediate values}
Some values must be provided by the user as \emph{input}. For example,
\cmd{main.c} can be edited by the user who relies on the build system to
compile it into \cmd{main.o} and subsequently \cmd{main.exe}. End build products,
such as \cmd{main.exe}, are \emph{output} values. All other values (in this case
\cmd{main.o}) are \emph{intermediate}; they are not interesting for the user
but are produced in the process of turning inputs into outputs.

% \vspace{-2mm}
% \paragraph{The store.}
A \emph{store}, of type \hs{Store}~\hs{i}~\hs{k}~\hs{v}, associates keys to
values. It is convenient to assume that a store is total, i.e. it contains a
value for every possible key. We therefore also assume that the type of values
is capable of encoding values corresponding to non-existent keys (missing files
or empty cells). The store also holds a value of type \hs{i},
representing the persistent build information maintained by the build
system~--~its ``memory''.
% Given an \hs{i} and a key/value map \store, we can \hs{initialise} a store.

\vspace{-2mm}
\paragraph{Hashes.}
We use a cryptographic \emph{hash function}
\hs{hash}~\hs{::}~\hs{Hashable}~\hs{v}~\hs{=>}~\hs{v}~\hs{->}~\hs{Hash}~\hs{v}
for efficient tracking and sharing of build results, where \hs{Hash}~\hs{v} is
an abstract data type equipped with~the~equality test.
In addition to usual \hs{getValue} and \hs{putValue} operations,
some stores support a dedicated \hs{getHash} operation, which is implemented
more efficiently than by hashing the result of \hs{getValue}\footnote{For
example, Git Virtual File System~\cite{gvfs} stores file hashes and downloads
file contents only on demand.}.

% This, in particular, suggests that it is possible to \emph{depend on the
% absence of a value}, which is a useful feature for build systems.

\subsection{The Task abstraction}\label{sec-task}

Our first main abstraction is for \emph{task descriptions}:
\begin{minted}[xleftmargin=10pt]{haskell}
type Task c k v = @\std{forall}@ f. c f => (k -> f v) -> k -> Maybe (f v)
\end{minted}
This highly-abstracted type\footnote{Readers familiar with \emph{lenses} or
\emph{profunctor optics} might recognise a familiar pattern. We discuss this
in~\S\ref{sec-related-optics}.} is best introduced by an example.
Consider this \Excel spreadsheet:
\vspace{1mm}
\begin{minted}[xleftmargin=10pt]{text}
A1: 10     B1: A1 + A2
A2: 20     B2: B1 * 2
\end{minted}
\vspace{1mm}
Here cell \cmd{A1} contains the value \cmd{10}, the cell \cmd{B1} contains
the formula \cmd{A1+A2}, etc. We can represent the formulae of this spreadsheet
with the following task description:
\vspace{1mm}
\begin{minted}[xleftmargin=10pt]{haskell}
sprsh1 :: Task Applicative String Integer
sprsh1 fetch "B1" = Just ((+)   <$> fetch "A1" <*> fetch "A2")
sprsh1 fetch "B2" = Just ((* 2) <$> fetch "B1")
sprsh1 _     _    = Nothing
\end{minted}
\vspace{1mm}
Here we instantiate keys \hs{k} with \hs{String}, and values \hs{v} with \hs{Integer}.
(Real spreadsheet cells would contain a wider range of values, of course.)
The task description \hs{sprsh1} embodies all the \emph{formulae} of the spreadsheet,
but not the input values.  Like every \hs{Task}, \hs{sprsh1} is given a
\emph{callback} \hs{fetch} and a key. It pattern-matches on the key to see if it
has a task description (in the \Excel case, a formula) for it. If not, it returns
\hs{Nothing}, indicating the key is an input. If there is a formula in the cell,
it computes the value of the formula, using \hs{fetch} to find the value of any
keys on which it depends.

The code to ``compute the value of a formula'' in \hs{sprsh1} looks a bit mysterious
because it takes place in an \hs{Applicative} computation \cite{mcbride2008applicative}
-- the relevant type signatures are given in Fig.~\ref{fig-types}. We will
explain why in subsection~\S\ref{sec-general-build}.

For now, we content ourselves with observing that a task description,
of type \hs{Task}~\hs{c}~\hs{k}~\hs{v}, is completely isolated from the world of
compilers, calculation chains, file systems, caches, and all other
complexities of real build systems.  It just computes a single output, in
a side-effect-free way, using a callback (\hs{fetch}) to find the values
of its dependencies.

\subsection{The Build abstraction}\label{sec-general-build}

Next comes our second main abstraction -- a build system:
\vspace{1mm}
\begin{minted}[xleftmargin=10pt]{haskell}
type Build c i k v = Task c k v -> k -> Store i k v -> Store i k v
\end{minted}
\vspace{1mm}
The signature is very straightforward.  Given a task description, a target key,
and a store, the build system returns a new store in which the value of the
target key is up to date. What exactly does ``up to date'' mean?  We answer
that precisely in \S\ref{sec-build-correctness}.

\noindent
Here is a simple build system:
\vspace{1mm}
\begin{minted}[xleftmargin=10pt]{haskell}
busy :: Eq k => Build Monad () k v
busy task key store = execState (fetch key) store
  where
    fetch :: k -> State (Store i k v) v
    fetch k = case task fetch k of
        Nothing  -> do s <- get; return (getValue s k)
        Just act -> do v <- act; modify (\s -> putValue s k v); return v
\end{minted}
\vspace{1mm}

\noindent
The \hs{busy} build system defines the callback \hs{fetch} so that, when given a
key, it brings the key up to date in the store, and returns its value.
The function \hs{fetch} runs in the standard Haskell \hs{State} monad -- see
Fig.~\ref{fig-types} -- initialised with the incoming \hs{store} by \hs{execState}.
To bring a key up to date, \hs{fetch} asks the task description \hs{task} how
to compute \hs{k}. If \hs{task} returns \hs{Nothing} the key is an input, so
\hs{fetch} simply reads the result from the store. Otherwise \hs{fetch} runs
the action \hs{act} returned by the \hs{task} to produce a resulting
value~\hs{v}, records the new key/value mapping in the store, and returns \hs{v}.
Notice that \hs{fetch} passes itself to \hs{task} as an argument, so that the
latter can use \hs{fetch} to recursively find the values of \hs{k}'s dependencies.

Given an acyclic task description, the \hs{busy} build system terminates with a
correct result, but it is not a \emph{minimal} build system
(Definition~\ref{def-minimal}). Since \hs{busy} has no memory
(\hs{i}~\hs{=}~\hs{()}), it cannot keep track of keys it has already built, and
will therefore busily recompute the same keys again and again if they have
multiple dependents. We will develop much more efficient build systems
in~\S\ref{sec-implementations}.

Nevertheless, \hs{busy} can easily handle the example \hs{sprsh1}
from the previous subsection~\S\ref{sec-task}. In the GHCi session below we
initialise the store with \cmd{A1} set to 10 and all other cells set to 20.

\begin{minted}[xleftmargin=10pt]{haskell}
@\ghci@ store  = initialise () (\key -> if key == "A1" then 10 else 20)
@\ghci@ result = busy sprsh1 "B2" store
@\ghci@ getValue result "B1"
30
@\ghci@ getValue result "B2"
60
\end{minted}

\noindent
As one can see, \hs{busy} built both \cmd{B2} and its dependency \cmd{B1} in the
right order (if it had built \cmd{B2} before building \cmd{B1}, the result would
have been $20 * 2 = 40$ instead of $(10 + 20) * 2 = 60$). As an example showing
that \hs{busy} is not minimal, imagine that the formula in cell \cmd{B2} was
\cmd{B1~+~B1} instead of \cmd{B1~*~2}. This would lead to calling
\hs{fetch}~\hs{"B1"} twice -- once per each occurrence of \cmd{B1} in the
formula.

\subsection{The need for polymorphism in \hs{Task}}\label{sec-why-polymorphism}

Now we can see why the \hs{Task} abstraction is polymorphic in \hs{f}:
\begin{minted}[xleftmargin=10pt]{haskell}
type Task c k v = @\std{forall}@ f. c f => (k -> f v) -> k -> Maybe (f v)
\end{minted}
The \hs{busy} build system instantiates \hs{f} to
\hs{State}~\hs{(Store}~\hs{i}~\hs{k}~\hs{v)},
so that \hs{fetch}~\hs{::}~\hs{k}~\hs{->}~\hs{f}~\hs{v} can side-effect the
\hs{Store}, thereby allowing successive calls to \hs{fetch} to communicate with
one another.

We really, really want \hs{Task} to be \emph{polymorphic} in \hs{f}.
Given \emph{one} task description \cmd{T}, we want to explore \emph{many} build
systems that can build \cmd{T} -- and we will do so in sections~\S\ref{sec-build}
and~\S\ref{sec-implementations}. As we shall see, each build system will use a
different \hs{f}, so the task description must not fix \hs{f}.

But nor can the task description work for \emph{any} \hs{f}; most task
descriptions (e.g. \hs{sprsh1} in \S\ref{sec-task}) require that \hs{f}
satisfies certain properties, such as \hs{Applicatove} or \hs{Monad}. That is
why \hs{Task} has the ``\hs{c}~\hs{f}~\hs{=>}'' constraint in its type,
expressing that \hs{f} can only be instantiated by types that satisfy the
constraint \hs{c}.

So the type \hs{Task} emerges naturally, almost inevitably.  But now that
it \emph{has} emgerged, we find the that constraints \hs{c} classify task descriptions
in a very interesting way:
\begin{itemize}
\item \hs{Task}~\hs{Applicative}. In \hs{sprsh1} we needed only \hs{Applicative}
  operations, expressing the fact that the dependencies between cells can be
  determined \emph{statically}; that is, by looking at the formulae, without
  ``computing'' them (see \S\ref{sec-deps}).
\item \hs{Task}~\hs{Monad}. As we shall see in \S\ref{sec-task-monad}, a monadic task
  description allows \emph{dynamic} dependencies, in which a formula may depend
  on the value of cell \cmd{C}, but \emph{which} cell \cmd{C} depends on the
  value of another cell \cmd{D}.
\item \hs{Task}~\hs{Functor} is somewhat degenerate: the task description cannot
  even use the application operator \hs{<*>}, which limits dependencies to a
  single linear chain. It is interesting to note that, when run on a
  \hs{Task}~\hs{Functor}, the \hs{busy} build system will build each key at most
  once, thus partially fulfilling the minimality requirement~\ref{def-minimal}.
  Alas, it still has no mechanism to decide which input keys changed since the
  previous build.
\item \hs{Task}~\hs{Alternative}, \hs{Task}~\hs{MonadPlus} and their
  variants can be used for describing tasks with a certain type of
  non-determinism, as discussed in~\S\ref{sec-non-determinism}.
\end{itemize}

Notice also that, even though \hs{busy} takes a \hs{Task}~\hs{Monad} as its
argument, an application of \hs{busy} to a \hs{Task}~\hs{Functor} or
a \hs{Task}~\hs{Applicative} will typecheck just fine. It feels a bit like
sub-typing, but it is actually just ordinary higher-rank polymorphism at
work~\cite{jones2007practical}.

\subsection{Monadic tasks}\label{sec-task-monad}

As explained in~\S\ref{sec-background-excel}, some task descriptions have dynamic
dependencies, which are determined by values of intermediate computations. In
our framework, such task descriptions correspond to the type
\hs{Task}~\hs{Monad}~\hs{k}~\hs{v}. Consider this spreadsheet example:

\vspace{1mm}
\begin{minted}[xleftmargin=10pt]{text}
A1: 10      B1: IF(C1=1,B2,A2)      C1: 1
A2: 20      B2: IF(C1=1,A1,B1)
\end{minted}

\noindent
Here, \cmd{B1} and \cmd{B2} statically depend on each other, but \Excel (which
uses dynamic dependencies) is perfectly happy. We can express this using
our task abstraction as follows:

% The spreadsheet example that uses
% the \hs{INDIRECT} function can be expressed very similarly: simply replace the
% line containing the \cmd{if} statement with \hs{fetch ("A" ++ show c1)}.

\vspace{1mm}
\begin{minted}[xleftmargin=10pt]{haskell}
sprsh2 :: Task Monad String Integer
sprsh2 fetch "B1" = Just $ do c1 <- fetch "C1"
                              if c1 == 1 then fetch "B2" else fetch "A2"
sprsh2 fetch "B2" = Just $ do c1 <- fetch "C1"
                              if c1 == 1 then fetch "A1" else fetch "B1"
sprsh2 _     _    = Nothing
\end{minted}
\vspace{1mm}

\noindent
The big difference compared to \hs{sprsh1} is that the computation now takes
place in a \hs{Monad}, which allows us to extract the value of \hs{c1} \emph{and
\hs{fetch} different keys depending on whether or not \hs{c1}~\hs{==}~\hs{1}}.

Since the \hs{busy} build system introduced in~\S\ref{sec-general-build} always
rebuilds every dependency it encounters, it is easy for it to handle dynamic
dependencies. For minimal build systems, however, dynamic dependencies, and hence
monadic tasks, are much more challenging, as we will see
in~\S\ref{sec-implementations}.

\subsection{Correctness of a build system} \label{sec-build-correctness}

We can now say what it means for a build system to be \emph{correct}, something
that is seldom stated formally. Our intuition is this: \emph{when the build
system completes, the specified key, and all its dependencies, should be up to
date}. What does ``up to date'' mean? It means that if we recompute the value of
the key (using the task description, and the final store), we should get exactly
the same value as we see in the final store.

To express this formally we need an auxiliary function \hs{compute}, that
computes the value of a key in a given store \emph{without attempting to update
any dependencies}:
\begin{minted}[xleftmargin=10pt]{haskell}
compute :: Task Monad k v -> Store i k v -> k -> Maybe v
compute task store key = runIdentity <$> task (Identity . getValue store) key
\end{minted}

\noindent
Here we do not need any effects in the \hs{fetch} callback to \hs{task}, so
we can use the standard Haskell \hs{Identity} monad (Fig.~\ref{fig-types}).
The use of \hs{Identity} just fixes the `impedance mismatch' between the pure
function \hs{getValue}~\hs{store}, whose type is \store, and the \hs{fetch}
argument of the \hs{task}, whose type must be \hs{k}~\hs{->}~\hs{f}~\hs{v} for
some \hs{f}. To fix the mismatch, we wrap the result of the pure function in the
\hs{Identity} monad: the function \hs{Identity}~\hs{.}~\hs{getValue}~\hs{store}
has the type \hs{k}~\hs{->}~\hs{Identity}~\hs{v}, and can now be passed to a
\hs{task}. The result comes as \hs{Maybe}~\hs{(Identity}~\hs{v)}, hence we now
need to get rid of the \hs{Identity} wrapper by applying \hs{runIdentity} to the
contents of \hs{Maybe}.

\vspace{-1mm}
\noindent\rule{\textwidth}{0.4pt}
\vspace{-7mm}
\definition[Correctness]{Suppose \hs{build} is a build system, \hs{task} is a
build task description, \hs{key} is an output key, \hs{store} is an initial
store, and \hs{result} is the store produced by running the build system with
parameters \hs{task}, \hs{key} and \hs{store}. Or, using the precise language of
our abstractions:

\vspace{1mm}
\begin{minted}[xleftmargin=10pt]{haskell}
build         :: Build c i k v
task          :: Task c k v
key           :: k
store, @@result :: Store i k v
result = @@build @@task @@key @@store
\end{minted}
\vspace{1mm}

\noindent
Then the \hs{result} is~\emph{correct} if there exists an \hs{ideal} store, such
that:

\begin{itemize}
    \item The \hs{ideal} store is \emph{consistent} with the \hs{task}, i.e.
    for all possible non-input keys \hs{k}, the result of recomputing the
    \hs{task} matches the value in the \hs{ideal} store:
    \vspace{-1mm}
    \[
    \hs{Just}~\hs{(}\hs{getValue}~\hs{ideal}~\hs{k)}~\hs{==}~\hs{compute}~\hs{task}~\hs{ideal}~\hs{k}.
    \]
    \item The \hs{result} and \hs{ideal} agree on the output \hs{key}, i.e.:
    \vspace{-1mm}
    \[
    \hs{getValue}~\hs{result}~\hs{key}~\hs{==}~\hs{getValue}~\hs{ideal}~\hs{key}.
    \]
    In other words, we require that the output \hs{key} has the correct value,
    but do not impose any restrictions on intermediate keys, therefore
    permitting shallow cloud builds.
    \item The \hs{store}, \hs{result} and \hs{ideal} agree on all input keys
    \hs{k} that belong to the transitive closure of \hs{key}'s dependencies:
    \[
    \,\,\,\,\,\,\,\hs{getValue}~\hs{store}~\hs{k}~\hs{==}~\hs{getValue}~\hs{result}~\hs{k}~\hs{&&}~\hs{getValue}~\hs{store}~\hs{k}~\hs{==}~\hs{getValue}~\hs{ideal}~\hs{k}
    \]
    This requirement asserts that (i) no inputs were corrupted during the
    build, and (ii)~the \hs{ideal} store has constraints not only on the output,
    but also on the inputs.
\end{itemize}
A build system is \emph{correct} if it produces a correct \hs{result} for any
given \hs{task}, \hs{key} and \hs{store}.
}
\label{def-correct}
\vspace{-2mm}
\rule{\textwidth}{0.4pt}

It is hard to satisfy the above definition of correctness given a task
description with cycles. All build systems discussed in this paper are correct
only under the assumption that the given task description is acyclic. This
includes the \hs{busy} build system introduced earlier: it will loop
indefinitely given a cyclic \hs{task}. Some build systems provide a limited
support for cyclic tasks, see~\S\ref{sec-iterative-compute}.

\subsection{Computing dependencies}\label{sec-deps}

Earlier we remarked that a \hs{Task}~\hs{Applicative} could only have static
dependencies. Usually we would extract such static dependencies by (in the case
of \Excel) looking at the syntax tree of the formula.  But a task description
has no such syntax tree. Yet, remarkably, we can use the polymorphism of a
\hs{Task}~\hs{Applicative} to find its dependencies \emph{without doing any of
the actual work}. Here is the code:
\vspace{1mm}
\begin{minted}[xleftmargin=10pt]{haskell}
dependencies :: Task Applicative k v -> k -> [k]
dependencies task key = case task (\k -> Const [k]) key of
                            Nothing         -> []
                            Just (Const ks) -> ks
\end{minted}
\vspace{1mm}
Here \hs{Const} is a standard Haskell type defined in Fig.~\ref{fig-types}. We
instantiate \hs{f} to \hs{Const}~\hs{[@@k]}.  So a value of type \hs{f}~\hs{v},
or in this case \hs{Const}~\hs{[@@k]}~\hs{v}, contains no value \hs{v}, but does
contain a list of keys of type \hs{[@@k]} which we use to record dependencies.
The \hs{fetch} callback that we pass to \hs{task} records a single dependency;
and the standard definition of \hs{Applicative} for \hs{Const} (which we give
in Fig.~\ref{fig-types}) combines the dependencies from different parts of the
task. Running the task with \hs{f}~=~\hs{Const}~\hs{[@@k]} will thus
accumulate a list of the task's dependencies -- and that is just what
\hs{dependencies} does:
\vspace{1mm}
\begin{minted}[xleftmargin=10pt]{haskell}
@\ghci@ dependencies sprsh1 "A1"
[@@]
\end{minted}
\begin{minted}[xleftmargin=10pt]{haskell}
@\ghci@ dependencies sprsh1 "B1"
["A1", "A2"]
\end{minted}
\vspace{1mm}

\noindent
Notice that these calls to \hs{dependencies} do no actual computation (in this
case, spreadsheet arithmetic). They cannot: we are not supplying a store or any
input numbers. So, through the wonders of polymorphism, we are able to extract
the dependencies of the spreadsheet formula, and to do so efficiently, simply by
running its code in a different \hs{Applicative}! This is not new, for example
see~\cite{free-applicatives}, but it is cool.

So much for applicative tasks. What about monadic tasks with dynamic
dependencies? As we have seen in~\S\ref{sec-background-shake}, dynamic
dependencies need to be tracked too. This cannot be done statically; notice that
the application of the function \hs{dependencies} to a \hs{Task}~\hs{Monad} will
not typecheck. We need to run a monadic task on a store with concrete values,
which will determine the discovered dependencies. Accordingly, we introduce
the function \hs{track}: a combination of \hs{compute} and \hs{dependencies}
that computes both the resulting value and the list of its dependencies:

\vspace{1mm}
\begin{minted}[xleftmargin=10pt]{haskell}
import Control.Monad.Writer
\end{minted}
\vspace{0.5mm}
\begin{minted}[xleftmargin=10pt]{haskell}
track :: Task Monad k v -> Store i k v -> k -> Maybe (v, [k])
track task store = runWriter . task (\k -> writer (getValue store k, [k]))
\end{minted}
\vspace{1mm}

\noindent
The implementation uses the standard Haskell \hs{Writer} monad, which allows us
to record addition information -- a list of keys of type \hs{[@@k]} -- when
computing a task. We instantiate \hs{f} to \hs{Writer}~\hs{[@@k]}~\hs{v}, and feed
\hs{fetch}~\hs{k}~\hs{=}~\hs{writer}~\hs{(@@getValue}~\hs{store}~\hs{k,}~\hs{[@@k])}
to the task, which in addition to fetching a value from the store, tracks the
corresponding key. Here is an example of \hs{track}ing monadic tasks:

\vspace{1mm}
\begin{minted}[xleftmargin=10pt]{haskell}
@\ghci@ store1 = initialise () (\key -> if key == "A1" then 10 else 1)
@\ghci@ track sprsh2 store1 "B1"
Just (1,["C1","B2"])
\end{minted}
\vspace{1mm}
\begin{minted}[xleftmargin=10pt]{haskell}
@\ghci@ store2 = initialise () (\key -> if key == "A1" then 10 else 20)
@\ghci@ track sprsh2 store2 "B2"
Just (20,["C1","B1"])
\end{minted}
\vspace{1mm}

\noindent
As expected, the dependencies of \hs{sprsh2} (see the spreadsheet
in~\S\ref{sec-task-monad}) are determined by the contents of the store.
Furthermore, the above example highlights the static cycle between cells
\cmd{B1} and \cmd{B2}: in \hs{store1}, we have \cmd{B1 = B2 = A1 = 10},
whereas in \hs{store2}, we have \cmd{B2 = B1 = A2 = 20}.

% \subsection{Free tasks}
% \todo{AM}{Introduce free constructions \hs{TaskA} and \hs{TaskM} giving
% intuitions on what tasks really are. This will take at least one page.}
%
% data TaskA k v a = InputA | TaskA [k] ([v] -> a)
%
% data TaskM k v a = InputM | DoneM a | TaskM [k] ([v] -> Task k v a)

% \subsection{Recognising an applicative task wearing a monadic hat}
% Although dynamic dependencies are very useful, they typically represent only a
% small fraction of all dependencies. One can exploit this in a build system by
% recognising applicative tasks even though they have the type
% \hs{Task}~\hs{Monad} and treating them specially.
