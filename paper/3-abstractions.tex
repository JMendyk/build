\section{Build systems, abstractly}\label{sec-abstractions}

\begin{figure}
\vspace{-1mm}
\begin{minted}[fontsize=\small]{haskell}
-- An abstract store containing a key/value map and persistent build information
data Store i k v -- i = info, k = key, v = value
initialise :: i -> (k -> v) -> Store i k v
getInfo    :: Store i k v -> i
putInfo    :: i -> Store i k v -> Store i k v
getValue   :: k -> Store i k v -> v
putValue   :: Eq k => k -> v -> Store i k v -> Store i k v
getHash    :: Hashable v => k -> Store i k v -> Hash v
hash       :: Hashable v => v -> Hash v -- provides a fast equality check
\end{minted}
\vspace{0mm}
\begin{minted}[fontsize=\small]{haskell}
-- Build tasks (see §3.2)
newtype Task  c k v = Task { run :: @\std{forall}@ f. c f => (k -> f v) -> f v }
type    Tasks c k v = k -> Maybe (Task c k v)
\end{minted}
\vspace{0mm}
\begin{minted}[fontsize=\small]{haskell}
-- Build system (see §3.3)
type Build c i k v = Tasks c k v -> k -> Store i k v -> Store i k v
\end{minted}
\vspace{0mm}
\begin{minted}[fontsize=\small]{haskell}
-- Build system components: a scheduler and a rebuilder (see §5)
type Scheduler c i ir k v = Rebuilder c ir k v -> Build c i k v
type Rebuilder c   ir k v = k -> v -> Task c k v -> Task (MonadState ir) k v
\end{minted}
\vspace{-3mm}
\caption{Type signatures of key build systems abstractions.}\label{fig-types}
\vspace{-3mm}
\end{figure}

This section presents purely functional abstractions that allow us to express
all the intricacies of build systems discussed in~\S\ref{sec-background},
and design complex build systems from simple primitives. Specifically, we
present the \emph{task} and \emph{build} abstractions in~\S\ref{sec-task}
and~\S\ref{sec-general-build}, respectively. Sections~\S\ref{sec-build}
and~\S\ref{sec-implementations} scrutinise the abstractions further and provide
concrete implementations for several build systems.

\vspace{-1.5mm}
\subsection{Common vocabulary for build systems}\label{sec-vocabulary}
\vspace{-0.5mm}
% We begin by establishing a common vocabulary for build systems:

\emph{Keys, values, and the store.} The goal of any build system is to
bring up to date a \emph{store} that implements a mapping from \emph{keys} to
\emph{values}. In software build systems the store is the file system, the
keys are filenames, and the values are file contents. In \Excel, the store is
the worksheets, the keys are cell names (such as \cmd{A1}) and the values are
numbers, strings, etc., displayed as the cell contents. Many build systems use
\emph{hashes} of values as compact summaries with a fast equality check.

\emph{Input, output, and intermediate values.} Some values must be provided by
the user as \emph{input}. For example, \cmd{main.c} can be edited by the user
who relies on the build system to compile it into \cmd{main.o} and subsequently
\cmd{main.exe}. End build products, such as \cmd{main.exe}, are \emph{output}
values. All other values (in this case \cmd{main.o}) are \emph{intermediate};
they are not interesting for the user but are produced in the process of turning
inputs into outputs.

\emph{Persistent build information.} As well as the key/value mapping, the
store also contains information maintained by the build system itself, which
persists from one invocation of the build system to the next -- its ``memory''.

\emph{Task description.} Any build system requires the user to specify how
to compute the new value for one key, using the (up to date) values of its
dependencies. We call this specification the \emph{task description}. For
example, in \Excel, the formulae of the spreadsheet constitute the task
description; in \Make the rules in the makefile are the task description.

\emph{Build system.} A \emph{build system} takes a task description, a
\emph{target} key, and a store, and returns a new store in which the target key
and all its dependencies have an up to date value.

We will model build systems concretely, as Haskell programs. To that end,
Fig.~\ref{fig-types} provides the type signatures for all key abstractions
introduced in the paper. For example, \hs{Store}~\hs{i}~\hs{k}~\hs{v} is the
type of stores, with several associate functions (\hs{getValue}, etc.). We use
\hs{k} as a type variable ranging over keys, \hs{v} for values, and \hs{i} for
the persistent build information. Fig.~\ref{fig-stdlib} lists standard library
definitions.

\subsection{The Task abstraction}\label{sec-task}

Our first main abstraction is for \emph{task descriptions}:
\begin{minted}[xleftmargin=10pt]{haskell}
newtype Task  c k v = Task { run :: @\std{forall}@ f. c f => (k -> f v) -> f v }
type    Tasks c k v = k -> Maybe (Task c k v)
\end{minted}

\noindent
Here \hs{c} stands for \emph{constraint}, such as \hs{Applicative} or \hs{Monad}.
A \hs{Task} describes a single build task, while \hs{Tasks} associates a
\hs{Task} to every non-input key; input keys are associated with
\hs{Nothing}. The highly-abstracted type\footnote{Readers
familiar with \emph{lenses} or \emph{profunctor optics} might recognise a
familiar pattern. We discuss this in~\S\ref{sec-related-optics}.} \hs{Task} is
best introduced by an example. Consider this \Excel spreadsheet:

\begin{minted}[xleftmargin=10pt]{text}
A1: 10     B1: A1 + A2
A2: 20     B2: B1 * 2
\end{minted}

\noindent
Here cell \cmd{A1} contains the value \cmd{10}, cell \cmd{B1} contains the
formula \cmd{A1+A2}, etc.

\begin{figure}
\begin{minted}[fontsize=\small]{haskell}
-- Applicative functors
pure  :: Applicative f => a -> f a
(<$>) :: Applicative f =>   (a -> b) -> f a -> f b -- Left-associative
(<*>) :: Applicative f => f (a -> b) -> f a -> f b -- Left-associative
\end{minted}
\vspace{1mm}
\begin{minted}[fontsize=\small]{haskell}
-- Standard State monad from Control.Monad.State
data State s a
instance Monad (State s)
get       :: State s s
gets      :: (s -> a) -> State s a
put       :: s -> State s ()
modify    :: (s -> s) -> State s ()
runState  :: State s a -> s -> (a, s)
execState :: State s a -> s -> s
\end{minted}
\vspace{1mm}
\begin{minted}[fontsize=\small]{haskell}
-- Standard types from Data.Functor.Identity and Data.Functor.Const
newtype Identity a = Identity { runIdentity :: a }
newtype Const m a  = Const    { getConst    :: m }
\end{minted}
\vspace{1mm}
\begin{minted}[fontsize=\small]{haskell}
instance Functor (Const m) where
    fmap _ (Const m) = Const m
\end{minted}
\vspace{1mm}
\begin{minted}[fontsize=\small]{haskell}
instance Monoid m => Applicative (Const m) where
    pure _              = Const mempty   -- mempty is the identity of the monoid m
    Const x <*> Const y = Const (x <> y) -- <> is the binary operation of the monoid m
\end{minted}
\vspace{-3mm}
\caption{Standard library definitions.}\label{fig-stdlib}
\vspace{-4mm}
\end{figure}

\noindent
We can represent the formulae of this spreadsheet with the following task
description:

\vspace{1mm}
\begin{minted}[xleftmargin=10pt]{haskell}
sprsh1 :: Tasks Applicative String Integer
sprsh1 "B1" = Just $ Task $ \fetch -> ((+)  <$> fetch "A1" <*> fetch "A2")
sprsh1 "B2" = Just $ Task $ \fetch -> ((*2) <$> fetch "B1")
sprsh1 _    = Nothing
\end{minted}
\vspace{1mm}

\noindent
We instantiate keys \hs{k} with \hs{String}, and values \hs{v} with \hs{Integer}.
(Real spreadsheet cells would contain a wider range of values, of course.) The
task description \hs{sprsh1} embodies all the \emph{formulae} of the spreadsheet,
but not the input values. It pattern-matches on the key to see if it has a task
description (in the \Excel case, a formula) for it. If not, it returns
\hs{Nothing}, indicating that the key is an input. If there is a formula in the
cell, it returns the \hs{Task} to compute the value of the formula. Every
\hs{Task} is given a \emph{callback} \hs{fetch} to find the value of any keys on
which it depends.

The code to ``compute the value of a formula'' in \hs{sprsh1} looks a bit
mysterious because it takes place in an \hs{Applicative}
computation~\cite{mcbride2008applicative} -- the relevant type signatures are
given in Fig.~\ref{fig-stdlib}. We will explain why in~\S\ref{sec-general-build}.

For now, we content ourselves with observing that a task description, of type
\hs{Tasks}~\hs{c}~\hs{k}~\hs{v}, is completely isolated from the world of
compilers, calc chains, file systems, caches, and all other complexities of real
build systems. It just computes a single output, in a side-effect-free way,
using a callback (\hs{fetch}) to find the values of its dependencies.

\subsection{The Build abstraction}\label{sec-general-build}

Next comes our second main abstraction -- a build system:

\vspace{1mm}
\begin{minted}[xleftmargin=10pt]{haskell}
type Build c i k v = Tasks c k v -> k -> Store i k v -> Store i k v
\end{minted}
\vspace{1mm}

\noindent
The signature is very straightforward. Given a task description, a target key,
and a store, the build system returns a new store in which the value of the
target key is up to date. What exactly does ``up to date'' mean?  We answer
that precisely in \S\ref{sec-build-correctness}. Here is a simple build system:

\vspace{1mm}
\begin{minted}[xleftmargin=10pt]{haskell}
busy :: Eq k => Build Applicative () k v
busy tasks key store = execState (fetch key) store
  where
    fetch :: k -> State (Store () k v) v
    fetch k = case tasks k of
        Nothing   -> gets (getValue k)
        Just task -> do v <- run task fetch; modify (putValue k v); return v
\end{minted}
\vspace{1mm}

\noindent
The \hs{busy} build system defines the callback \hs{fetch} so that, when given a
key, it brings the key up to date in the store, and returns its value.
The function \hs{fetch} runs in the standard Haskell \hs{State} monad -- see
Fig.~\ref{fig-stdlib} -- initialised with the incoming \hs{store} by \hs{execState}.
To bring a key up to date, \hs{fetch} asks the task description \hs{tasks} how
to compute the value of \hs{k}. If \hs{tasks} returns \hs{Nothing} the key is an
input, so \hs{fetch} simply reads the result from the store. Otherwise \hs{fetch}
runs the obtained \hs{task} to produce a resulting value~\hs{v}, records the new
key/value mapping in the store, and returns \hs{v}. Notice that \hs{fetch}
passes itself to \hs{task} as an argument, so that the latter can use \hs{fetch}
to recursively find the values of \hs{k}'s dependencies.

Given an acyclic task description, the \hs{busy} build system terminates with a
correct result, but it is not a \emph{minimal} build system
(Definition~\ref{def-minimal}). Since \hs{busy} has no memory
(\hs{i}~\hs{=}~\hs{()}), it cannot keep track of keys it has already built, and
will therefore busily recompute the same keys again and again if they have
multiple dependents. We will develop much more efficient build systems
in~\S\ref{sec-implementations}.

Nevertheless, \hs{busy} can easily handle the example \hs{sprsh1}
from the previous subsection~\S\ref{sec-task}. In the GHCi session below we
initialise the store with \cmd{A1} set to 10 and all other cells set to 20.

\begin{minted}[xleftmargin=10pt]{haskell}
@\ghci@ store  = initialise () (\key -> if key == "A1" then 10 else 20)
@\ghci@ result = busy sprsh1 "B2" store
@\ghci@ getValue "B1" result
30
@\ghci@ getValue "B2" result
60
\end{minted}

\noindent
As we can see, \hs{busy} built both \cmd{B2} and its dependency \cmd{B1} in the
right order (if it had built \cmd{B2} before building \cmd{B1}, the result would
have been $20 * 2 = 40$ instead of $(10 + 20) * 2 = 60$). As an example showing
that \hs{busy} is not minimal, imagine that the formula in cell \cmd{B2} was
\cmd{B1~+~B1} instead of \cmd{B1~*~2}. This would lead to calling
\hs{fetch}~\hs{"B1"} twice -- once per occurrence of \cmd{B1} in the
formula.

\subsection{The need for polymorphism in \hs{Task}}\label{sec-why-polymorphism}
\vspace{-0.5mm}

The previous example shows why the \hs{Task} abstraction is polymorphic in \hs{f}, recall the definition:

\begin{minted}[xleftmargin=10pt]{haskell}
newtype Task c k v = Task { run :: @\std{forall}@ f. c f => (k -> f v) -> f v }
\end{minted}

\noindent
The \hs{busy} build system instantiates \hs{f} to
\hs{State}~\hs{(Store}~\hs{i}~\hs{k}~\hs{v)},
so that \hs{fetch}~\hs{::}~\hs{k}~\hs{->}~\hs{f}~\hs{v} can side-effect the
\hs{Store}, thereby allowing successive calls to \hs{fetch} to communicate with
one another.

We really, really want \hs{Task} to be \emph{polymorphic} in \hs{f}.
Given \emph{one} task description \cmd{T}, we want to explore \emph{many} build
systems that can build \cmd{T} -- and we will do so in sections~\S\ref{sec-build}
and~\S\ref{sec-implementations}. As we shall see, each build system will use a
different \hs{f}, so the task description must not fix \hs{f}.

But nor can the task description work for \emph{any} \hs{f}; most task
descriptions (e.g. \hs{sprsh1} in \S\ref{sec-task}) require that \hs{f}
satisfies certain properties, such as \hs{Applicative} or \hs{Monad}. That is
why \hs{Task} has the ``\hs{c}~\hs{f}~\hs{=>}'' constraint in its type,
expressing that \hs{f} can only be instantiated by types that satisfy the
constraint \hs{c}. So the type \hs{Task} emerges naturally, almost inevitably.
But now that it \emph{has} emerged, we find that constraints \hs{c} classify
task descriptions in a very interesting way:
\begin{itemize}
\item \hs{Task}~\hs{Applicative}. In \hs{sprsh1} we needed only \hs{Applicative}
  operations, expressing the fact that the dependencies between cells can be
  determined \emph{statically}; that is, by looking at the formulae, without
  ``computing'' them (see \S\ref{sec-deps}).
\item \hs{Task}~\hs{Monad}. As we shall see in \S\ref{sec-task-monad}, a monadic
  task allows \emph{dynamic} dependencies, in which a formula may depend on cell
  \cmd{C}, but \emph{which} cell \cmd{C} depends on the value of another cell
  \cmd{D}.
%TODO: Add a reference for Docker?
\item \hs{Task}~\hs{Functor} is somewhat degenerate: the task description cannot
  even use the application operator \hs{<*>}, which limits dependencies to a
  linear chain, as e.g. in Docker containers~\cite{docker}. It is
  interesting to note that, when run on such a task description, the \hs{busy}
  build system will build each key at most once, thus partially fulfilling the
  minimality requirement~\ref{def-minimal}. Alas, it still has no mechanism to
  decide which input keys changed since the previous build.
\item \hs{Task}~\hs{Alternative}, \hs{Task}~\hs{MonadPlus} and their
  variants can be used for describing tasks with a certain type of
  non-determinism, as discussed in~\S\ref{sec-non-determinism}.
\item \hs{Task}~\hs{(MonadState}~\hs{i)} will be used
  in~\S\ref{sec-implementations} to describe tasks that have read and write
  access to the persistently stored build information~\hs{i}.
\end{itemize}

% TODO: Shall we describe the lifTask and liftTasks combinators? They aren't pretty.
%
%   liftTask :: Task Applicative k v -> Task Monad k v
%   liftTask (Task task) = Task task
%
%   liftTasks :: Tasks Applicative k v -> Tasks Monad k v
%   liftTasks = fmap (fmap liftTask)
%
% Notice also that, even though \hs{busy} takes a \hs{Task}~\hs{Monad} as its
% argument, an application of \hs{busy} to a \hs{Task}~\hs{Functor} or
% a \hs{Task}~\hs{Applicative} will typecheck and run just fine. It feels a bit like
% sub-typing, but is actually just ordinary higher-rank polymorphism at
% work~\cite{jones2007practical}.

\vspace{-0.5mm}
\subsection{Monadic tasks}\label{sec-task-monad}

As explained in~\S\ref{sec-background-excel}, some task descriptions have
dynamic dependencies, which are determined by values of intermediate
computations. In our framework, such task descriptions correspond to the type
\hs{Tasks}~\hs{Monad}~\hs{k}~\hs{v}. Consider this spreadsheet example:

\vspace{0.5mm}
\begin{minted}[xleftmargin=10pt]{text}
A1: 10      B1: IF(C1=1,B2,A2)      C1: 1
A2: 20      B2: IF(C1=1,A1,B1)
\end{minted}
\vspace{0.5mm}

\noindent
Note that \cmd{B1} and \cmd{B2} statically form a dependency cycle, but \Excel
(which uses dynamic dependencies) is perfectly happy. We can express this
spreadsheet using our task abstraction as:

% The spreadsheet example that uses
% the \hs{INDIRECT} function can be expressed very similarly: simply replace the
% line containing the \cmd{if} statement with \hs{fetch ("A" ++ show c1)}.

\vspace{0.5mm}
\begin{minted}[xleftmargin=10pt]{haskell}
sprsh2 :: Tasks Monad String Integer
sprsh2 "B1" = Just $ Task $ \fetch -> do
    c1 <- fetch "C1"
    if c1 == 1 then fetch "B2" else fetch "A2"
sprsh2 "B2" = Just $ Task $ \fetch -> do
    c1 <- fetch "C1"
    if c1 == 1 then fetch "A1" else fetch "B1"
sprsh2 _ = Nothing
\end{minted}
\vspace{0.5mm}

\noindent
The big difference compared to \hs{sprsh1} is that the computation now takes
place in a \hs{Monad}, which allows us to extract the value of \hs{c1} and
\hs{fetch} \emph{different keys depending on whether or not \hs{c1}~\hs{==}~\hs{1}}.

Since the \hs{busy} build system introduced in~\S\ref{sec-general-build} always
rebuilds every dependency it encounters, it is easy for it to handle dynamic
dependencies. For minimal build systems, however, dynamic dependencies, and hence
monadic tasks, are much more challenging, as we shall see
in~\S\ref{sec-implementations}.

\subsection{Correctness of a build system} \label{sec-build-correctness}

We can now say what it means for a build system to be \emph{correct}, something
that is seldom stated formally. Our intuition is this: \emph{when the build
system completes, the target key, and all its dependencies, should be up to
date}. What does ``up to date'' mean? It means that if we recompute the value of
the key (using the task description, and the final store), we should get exactly
the same value as we see in the final store.

To express this formally we need an auxiliary function \hs{compute}, that
computes the value of a key in a given store \emph{without attempting to update
any dependencies}:

\begin{minted}[xleftmargin=5pt]{haskell}
compute :: Task Monad k v -> Store i k v -> v
compute task store = runIdentity $ run task (\k -> Identity (getValue k store))
\end{minted}

\noindent
Here we do not need any effects in the \hs{fetch} callback to \hs{task}, so
we can use the standard Haskell \hs{Identity} monad (Fig.~\ref{fig-stdlib}).
The use of \hs{Identity} just fixes the `impedance mismatch' between the
function \hs{getValue}, which returns a pure value~\hs{v}, and the \hs{fetch}
argument of the \hs{task}, which must return an \hs{f}~\hs{v} for some \hs{f}.
To fix the mismatch, we wrap the result of \hs{getValue} in the \hs{Identity}
monad: the function
\hs{\@@k}~\hs{->}~\hs{Identity}~\hs{(@@getValue}~\hs{k}~\hs{store)} has the type
\hs{k}~\hs{->}~\hs{Identity}~\hs{v}, and can now be passed to a \hs{task}. The
result comes as \hs{Identity}~\hs{v}, which we unwrap by \hs{runIdentity}.

\vspace{-1mm}
\definition[Correctness]{Suppose \hs{build} is a build system, \hs{tasks} is a
build task description, \hs{key} is a target key, \hs{store} is an initial
store, and \hs{result} is the store produced by running the build system with
parameters \hs{tasks}, \hs{key} and \hs{store}. Or, using the precise language of
our abstractions:

\begin{minted}[xleftmargin=10pt]{haskell}
build         :: Build c i k v
tasks         :: Tasks c k v
key           :: k
store, @@result :: Store i k v
result = @@build @@tasks @@key @@store
\end{minted}

\noindent
The keys that are reachable from the target \hs{key} via dependencies fall
into two classes: input keys and non-input keys, which we will denote by $I$ and
$O$, respectively. Note that \hs{key} may be in either of these sets, although
the case when \hs{key} is an input is degenerate: we have $I=\{$\hs{key}$\}$ and
$O=\emptyset$.

The build \hs{result} is~\emph{correct} if the following two conditions hold:

\begin{itemize}
    \item \hs{result} and \hs{store} \emph{agree on inputs}, that is, for all
    input keys \hs{k}~$\in$~$I$:
    \[
    \hs{getValue}~\hs{k}~\hs{result}~\hs{==}~\hs{getValue}~\hs{k}~\hs{store}.
    \]
    In other words, no inputs were corrupted during the
    build.
    %\footnote{\todo{SLPJ}{Add the joke that GHC used to delete inputs?}}.

    \item The \hs{result} is \emph{consistent} with the \hs{tasks}, i.e.
    for all non-input keys \hs{k}~$\in$~$O$, the result of recomputing the
    corresponding \hs{task} matches the value stored in the \hs{result}:
    \[
    \hs{getValue}~\hs{k}~\hs{result}~\hs{==}~\hs{compute}~\hs{task}~\hs{result}.
    \]
\end{itemize}
A build system is \emph{correct} if it produces a correct \hs{result} for any
given \hs{tasks}, \hs{key} and \hs{store}.
}
\label{def-correct}
\vspace{2mm}

It is hard to satisfy the above definition of correctness given a task
description with cycles. All build systems discussed in this paper are correct
only under the assumption that the given task description is acyclic. This
includes the \hs{busy} build system introduced earlier: it will loop
indefinitely given a cyclic \hs{tasks}. Some build systems provide a limited
support for cyclic tasks, see~\S\ref{sec-iterative-compute}.

The presented definition of correctness needs to be adjusted to accommodate
non-deterministic tasks and shallow cloud builds, as will be discussed
in sections~\S\ref{sec-non-determinism} and~\S\ref{sec-cloud-aspects},
respectively.

\subsection{Computing dependencies}\label{sec-deps}\label{secdeps}

Earlier we remarked that a \hs{Task}~\hs{Applicative} could only have static
dependencies. Usually we would extract such static dependencies by (in the case
of \Excel) looking at the syntax tree of the formula.  But a task description
has no such syntax tree. Yet, remarkably, we can use the polymorphism of a
\hs{Task}~\hs{Applicative} to find its dependencies \emph{without doing any of
the actual work}. Here is the code:

\vspace{1mm}
\begin{minted}[xleftmargin=10pt]{haskell}
dependencies :: Task Applicative k v -> [k]
dependencies task = getConst $ run task (\k -> Const [k])
\end{minted}
\vspace{1mm}

\noindent
Here \hs{Const} is a standard Haskell type defined in Fig.~\ref{fig-stdlib}. We
instantiate \hs{f} to \hs{Const}~\hs{[@@k]}. So a value of type \hs{f}~\hs{v},
or in this case \hs{Const}~\hs{[@@k]}~\hs{v}, contains no value \hs{v}, but does
contain a list of keys of type \hs{[@@k]} which we use to record dependencies.
The \hs{fetch} callback that we pass to \hs{task} records a single dependency;
and the standard definition of \hs{Applicative} for \hs{Const} (which we give
in Fig.~\ref{fig-stdlib}) combines the dependencies from different parts of the
task. Running the task with \hs{f}~=~\hs{Const}~\hs{[@@k]} will thus
accumulate a list of the task's dependencies -- and that is just what
\hs{dependencies} does:
\vspace{1mm}
\begin{minted}[xleftmargin=10pt]{haskell}
@\ghci@ dependencies $ fromJust $ sprsh1 "A1"
[@@]
\end{minted}
\begin{minted}[xleftmargin=10pt]{haskell}
@\ghci@ dependencies $ fromJust $ sprsh1 "B1"
["A1", "A2"]
\end{minted}
\vspace{1mm}

\noindent
Notice that these calls to \hs{dependencies} do no actual computation (in this
case, spreadsheet arithmetic). They cannot: we are not supplying a store or any
input numbers. So, through the wonders of polymorphism, we are able to extract
the dependencies of the spreadsheet formula, and to do so efficiently, simply by
running its code in a different \hs{Applicative}! This is not new, for example
see~\citet{free-applicatives}, but it is extremely cool. We will see a practical
use for \hs{dependencies} when implementing applicative build systems,
see~\S\ref{sec-implementation-make}.

So much for applicative tasks. What about monadic tasks with dynamic
dependencies? As we have seen in~\S\ref{sec-background-shake}, dynamic
dependencies need to be tracked too. This cannot be done statically; notice that
we cannot apply the function \hs{dependencies} to a \hs{Task}~\hs{Monad} because
the \hs{Const} functor has no \hs{Monad} instance. We need to run a monadic task
on a store with concrete values, which will determine the discovered
dependencies. Accordingly, we introduce the function \hs{track}: a combination
of \hs{compute} and \hs{dependencies} that computes both the resulting value and
the list of its dependencies (key/value pairs) in an arbitrary monadic
context~\hs{m}:

\vspace{1mm}
\begin{minted}[xleftmargin=10pt]{haskell}
import Control.Monad.Writer
\end{minted}
\vspace{0.5mm}
\begin{minted}[xleftmargin=10pt]{haskell}
track :: Monad m => Task Monad k v -> (k -> m v) -> m (v, [(k, v)])
track task fetch = runWriterT $ run task trackingFetch
  where
    trackingFetch :: k -> WriterT [(k, v)] m v
    trackingFetch k = do v <- lift (fetch k); tell [(k, v)]; return v
\end{minted}
\vspace{1mm}

\noindent
This implementation uses the standard Haskell \hs{WriterT} \emph{monad
transformer}~\cite{liang1995monad}, which allows us to record additional
information -- a list of key/value pairs \hs{[(@@k,}~\hs{v)]} -- when computing
a task in an arbitrary monad~\hs{m}. We substitute the given \hs{fetch} with a
\hs{trackingFetch} that, in addition to fetching a value, tracks the
corresponding key/value pair. The \hs{task} returns the value of type
\hs{WriterT}~\hs{[(@@k,}~\hs{v)]}~\hs{m}~\hs{v}, which we unwrap by
\hs{runWriterT}. We will use \hs{track} when implementing monadic build systems
with dynamic dependencies, see~\S\ref{sec-implementation-shake}. Here we show an
example of \hs{track}ing monadic tasks when \hs{m}~\hs{=}~\hs{IO}.

\begin{minted}[xleftmargin=10pt]{haskell}
@\ghci@ fetchIO k = do putStr (k ++ ": "); read <$> getLine
@\ghci@ track (fromJust $ sprsh2 "B1") fetchIO
C1: 1
B2: 10
(10,[("C1",1),("B2",10)])
\end{minted}
\vspace{1mm}
\begin{minted}[xleftmargin=10pt]{haskell}
@\ghci@ track (fromJust $ sprsh2 "B1") fetchIO
C1: 2
A2: 20
(20,[("C1",2),("A2",20)])
\end{minted}
\vspace{1mm}

\noindent
As expected, the dependencies of the cell \cmd{B1} from \hs{sprsh2} (see the
spreadsheet in~\S\ref{sec-task-monad}) are determined by the value of \cmd{C1},
which in this case is obtained by reading from the standard input.
