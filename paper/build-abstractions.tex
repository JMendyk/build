\clearpage

\section{Build systems, abstractly}\label{sec-build-abstractions}

In this section we present the \emph{compute} and \emph{build} abstractions:
\begin{itemize}
    \item Compute represents build rules (in build systems) or formulas (in
    spreadsheets), and is completely isolated from the world of compilers, file
    systems, memories, caches, and all other complexities of real build systems.
    \item Build corresponds to the algorithm used for bringing a key-value store
    to a coherent state by running compute.
\end{itemize}

\begin{figure}
\begin{minted}{haskell}
type Compute f k v = (k -> f v) -> k -> f (Maybe v)

type FunctorialCompute  f k v = Functor     f => Compute f k v
type ApplicativeCompute f k v = Applicative f => Compute f k v
type AlternativeCompute f k v = Alternative f => Compute f k v
type MonadicCompute     f k v = Monad       f => Compute f k v
type MonadPlusedCompute f k v = MonadPlus   f => Compute f k v
\end{minted}
\caption{Compute abstractions}\label{fig-compute}
\end{figure}

Consider abstractions in Fig.~\ref{fig-compute}.

...

We now describe several types of build systems according to their constraint
on the functor \hs{f}.

...

