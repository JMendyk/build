\clearpage
\section{Build systems, concretely}\label{sec-implementations}

In the previous sections we discussed the types of build system, and how they broken down. But these divisions were not obvious to us, and only by concretely implementing and refactoring each build system did we determine the underlying commonalities. In this section we share some of the code that got us there, but full code is available in the supporting material and at \url{url-hidden-for-anonymous-submission}. Our code comprises a framework, utility functions implementing the various dependency orderings as per \S\ref{sec-dependency-orderings}, and actual implementations.

\subsection{Framework}

As we saw in \S\ref{sec-build}, a build system can be defined as:

\begin{minted}{haskell}
type Build c i k v = Compute c k v -> k -> Maybe i -> (k -> v) -> (i, k -> v)
\end{minted}

Using the framework we build in this section, we can express a dumb build system as:

\begin{minted}{haskell}
dumb :: Build Monad () k v
dumb compute k = runM (f k)
    where
        f k = case compute f k of
            Nothing -> getStore k
            Just act -> do
                v <- act
                putStore k v
                return v
\end{minted}

The \hs{dumb} build system can build with monadic computes, using a function \hs{f} which given a key returns the associated value. To do so, \hs{f} calls \hs{compute} passing \hs{f} itself as an argument. If the result is \hs{Nothing} (this key is an input) it simply reads from the store. Otherwise it runs the action to produce a value, stores it and returns it.

\begin{figure}
\begin{minted}{haskell}
type M i k v r = State (i, k -> v, ...) r
runM :: Default i => M i k v a -> Maybe i -> (k -> v) -> (i, k -> v)

getStore :: k -> M i k v v
putStore :: k -> v -> M i k v ()
getInfo :: M i k v i
modifyInfo :: (i -> i) -> M i k v ()
\end{minted}
Note: we have omitted the context on \hs{k}, things like \hs{Eq} (to implement the store).
\caption{API with which to implement build systems}
\label{fig-M-api}
\end{figure}

We use \hs{runM} to switch between the pure world of \hs{Build} and a state monad \hs{M} which lets us more precisely express the ideas behind the build system without threading the state manually - the types of important operations are given in Figure~\ref{fig-M-api}. We provide functions like \hs{getStore} to load from the store and \hs{putStore} to write to the store. We'll introduce the other functions in in the monad \hs{M} as we need them while exploring the build systems.

\todo{AM}{AM: What about using shorter \hs{get} and \hs{put}? I can introduce it right in the common vocabulary, mentioning the usual get-put laws. I would also remove `get' prefixes from many functions, e.g. \hs{dependencies compute} reads fine. I suggest not to rush with this TODO -- let me first finish the Compute section, where many of these helper functions will be defined. NM: I think this is a bad idea - conflicts with a state Monad which is what we're using underneath so will confuse some readers.}

\subsection{\Make}\label{sec-implementation-make}

The \Make build system builds files in a linear order based on a topological sort. For each file, it builds it if the output file is older than any of the inputs. We can express this in our framework with:

\begin{minted}{haskell}
make :: Eq v => Build Applicative (Changed k v, ()) k v
make = withChanged $ topological $ \k dk act -> do
    t <- getStoreTime k
    dt <- mapM getStoreTime dk
    let dirty = any (>= t) dt
    when dirty $
        putStore k =<< act
\end{minted}

We'll come back to \hs{withChanged} afterwards, but ignoring that, reading the rest of the implementation you can see the underying essence of \Make. We use the helper function \hs{topological} which gets called with each file in a dependency-respecting order given the file (\hs{k}), the dependencies (\hs{dk}) and an action to generate the result (\hs{act}). With those three things in hand, the implementation first gets the timestamp of the key being built, then the timestamps of the dependencies (which must have been rebuilt if necessary, thanks to the guarantees on \hs{topological}), and considers it dirty if any dependency was built after \hs{k}. If it is dirty, it rebuilds and stores the result.

The implementation of \hs{topological} encodes the dependency strategy that \Make has chosen to use. The function itself is defined as:

\begin{minted}{haskell}
topological
    :: (m ~ M i k v, Default i)
    => (k -> [k] -> m v -> m ())
    -> Build Applicative i k v
topological step compute k = runM $ do
    let depends = getDependencies compute
    forM_ (topSort depends $ transitiveClosure depends k) $ \k ->
        case compute getStore k of
            Nothing -> return ()
            Just act -> step k (depends k) act
\end{minted}

It uses a \hs{getDependencies} on the computation function, takes the transitive-closure and topological-sort to provide an order in which to build the keys. For each key, if it is an input it does nothing, otherwise it calls the supplied \hs{step} function. Note that \hs{getDependencies} is only defined for \hs{Applicative} computes, which is what restricts \Make to Applicative build systems. Moreover, anyone following the same topological-sort approach will also have the \hs{Applicative} restriction.

The greatest impedence mismatch between our framework and this implementation is the notion of \hs{Time} in \hs{getStoreTime}, which is missing from \hs{Build}. However, as we discussed in \S\ref{sec-dirty-bit}, \Make is really using modification-time to encode a dirty bit with specific properties. We model the dirty bit and time with the following API on \hs{M}:

\begin{minted}{haskell}
getChanged :: Eq v => k -> M (Changed k v, i) k v Bool
withChanged :: Build c (Changed k v, i) k v -> Build c (Changed k v, i) k v

data Time = LastBuild | AfterLastBuild
getStoreTime :: Eq v => k -> M (Changed k v, i) k v Time
\end{minted}

To model a dirty bit we extend the \hs{i} information with \hs{Changed} to capture the \hs{k -> v} store at the end of the previous run, and extend \hs{M} to record which files have been written to during this \hs{runM}. A file is considered dirty if either the current value as per \hs{getStore} has changed since the last run, or if it has had \hs{putStore} called on it\footnote{It would be reasonable to declare files unchanged if they have \hs{putStore} called at their previous value, but \Excel does not and \Make cannot using modification times, so we ignore this option.}. To track the values at the last run we use \hs{withChanged} to copy the resultant \hs{k -> v} store into the info after the build completes\footnote{For tedious reasons our implementation requires \hs{withChangedApplicative} and \hs{withChangedMonad} which are variants of \hs{withChanged} for concrete constraints \hs{c} - but with identical implementations.}

On top of a dirty bit we can encode a time with a very small domain, concretely if the file was last changed before the last build ended, or afterwards. That domain is sufficient for the \Make system. At first the fact that this domain is sufficient surprised us, until we realised \Make is only interested in the dirty bit anyway! If we strip back the modification time and directly implement \Make in terms of \hs{getChanged} we obtain:

\begin{minted}{haskell}
makeDirtyBit :: Eq v => Build Applicative (Changed k v, ()) k v
makeDirtyBit = withChanged $ topological $ \k dk act -> do
    dirty <- getChanged k ||^ anyM getChanged dk
    when dirty $
        putStore k =<< act
\end{minted}

\subsection{\Excel}\label{sec-implementation-excel}

As a programming exercise, \Excel is the most challenging, due to the complicated reordering dependency scheme, which requires retrying out-of-order operations. The complete implementation is:

\begin{minted}{haskell}
excel :: Eq v => Build Monad (Changed k v, [k]) k v
excel = withChanged $ reordering $ \k dk _ act -> do
    dirty <- getChanged k ||^ maybe (return True) (anyM getChanged) dk
    if not dirty then
        return Nothing
    else
        act >>= \case
            Left e -> return $ Just e
            Right (_, v) -> do
                putStore k v
                return Nothing
\end{minted}

We use the helper function \hs{reordering} which provides the key to build (\hs{k}), the dependencies of \hs{k} (\hs{dk}) and an action to build \hs{k} (\hs{act}). The dirty check is the same as \hs{makeDirtyBit}, with the added complication that because the \hs{Compute} is monadic it may not be possible to statically get a list of dependencies - if \hs{dk} is \hs{Nothing} then the action uses the monadic power and we conservatively rebuild anyway. If the check determines that the key is dirty, and needs rebuilding, we run \hs{act}, giving two possiblities:

\begin{enumerate}
\item The key cannot be built at this time because it depends on a key which has not yet been computed - and thus \hs{act} returns \hs{Left e}, where \hs{e} is the key that we are waiting for. In response we return \hs{e} as the result of step - asking \hs{reordering} to requeue this key after \hs{e}.
\item The key builds successfully, so we store it and return \hs{Nothing} to indicate \hs{reordering} can move on to the next key.
\end{enumerate}

The astute reader may notice that \hs{reordering} supplies extra information that is ignored with \hs{_} patterns. While not required for \Excel, we have implemented build systems using \hs{reordering} which use verifying and constructive traces, effectively turning \Excel into a cloud build system and ensuring \hs{reordering} is not overly fitted to \Excel alone (additional implementations available in the suplementary material). The definition of \hs{reordering} is:

\begin{minted}{haskell}
reordering
    :: (m ~ M (i, [k]) k v, Default i)
    => (k -> Maybe [k] -> (k -> m (Maybe v)) -> m (Either k ([k], v)) -> m (Maybe k))
    -> Build Monad (i, [k]) k v
\end{minted}

Dynamic requires a function that takes takes 4 arguments (we use \hs{m} to elide \hs{M (i, [k]) k v} from all the signatures):

\begin{enumerate}
\item \hs{k} - the key that is being built.
\item \hs{Maybe [k]} - the list of dependencies, or \hs{Nothing} if the step makes use of monadic dependencies. Note that above if a step is truely monadic it is always rebuilt. In the case of spreadsheets such steps are very rare, so a very conservative approximation can be used.\footnote{In truth, many spreadsheets take the alternative approximation of always assuming monadic steps \textit{did not} change - trading correctness for performance.}
\item \hs{k -> m (Maybe v)} - a function that looks up a key - it returns \hs{Nothing} if that key has not yet been built (and thus there is a dependency violation).
\item \hs{m (Either k ([k], v))} - an action to produce either the \hs{k} which is required but not yet computed (the reason you fail your dependency order check), or a list of the keys used and value produced.
\item It returns \hs{m (Maybe k)}, with a \hs{Nothing} to indicate the the build succeeded, or \hs{Just k} to indicate that the given key \hs{k} has not been built, and thus this key should be moved after \hs{k}.
\end{enumerate}

The implementation of \hs{dynamic} is 15 lines, but is substantially complicated by the requirement to ensure newly discovered keys are added to the list - something an actual spreadsheet with a fixed universe of cells need not do.

\subsection{\Shake}\label{sec-implementation-shake}

The Shake approach for dependency tracking involves recording traces and verifying them. Using the \hs{Trace} type defined in \ref{sec-verifying-traces} we can implement:

\begin{minted}{haskell}
shake :: Hashable v => Build Monad [Trace k v] k v
shake = recursive $ \k _ ask act -> do
    poss <- traceMatch (\k v -> (==) v . getHash <$> ask k) k =<< getInfo
    h <- getStoreHash k
    when (h `notElem` poss) $ do
        (dk, v) <- act
        putStore k v
        dh <- mapM getStoreHash dk
        modifyInfo (Trace k (zip dk dh) (getHash v) :)
\end{minted}

This implementation uses the \hs{recursive} function which requires a function which gets the key to build (\hs{k}), a way to ask for the result of a dependecy (\hs{ask}), and a way to build \hs{k} (\hs{act}). The code first looks up the traces that match the current situation, creating a list of possible results as \hs{poss}, using the helper \hs{traceMatch}. We check if the current value for the key is one of the possible results, and it not, build the key and store it in the trace. We can implement \hs{traceMatch}, which is used in all build systems working with \hs{Trace}, as:

\begin{minted}{haskell}
traceMatch :: (Monad m, Eq k) => (k -> Hash v -> m Bool) -> k -> [Trace k v] -> m [Hash v]
traceMatch check k ts = mapMaybeM f ts
    where f (Trace k2 dkv v) = do
                b <- return (k == k2) &&^ allM (uncurry check) dkv
                return $ if b then Just v else Nothing
\end{minted}

The function \hs{traceMatch} requires a way to check a dependency, and the key, and the traces. We return the \hs{result} field from all traces where the key and dependencies match. We define the recursive combinator as:

\begin{minted}{haskell}
recursive
    :: (m ~ M i k v, Default i)
    => (k -> Maybe [k] -> (k -> m v) -> m ([k], v) -> m ())
    -> Build Monad i k v
recursive step compute = runM . ensure
    where
        ensure k = do
            let ask x = ensure x >> getStore x
            done <- getTemp
            when (k `Set.notMember` done) $ do
                modifyTemp $ Set.insert k set
                case trackDependencies compute ask k of
                    Nothing -> return ()
                    Just act -> step k (getDependenciesMaybe compute k) ask act
\end{minted}

The required \hs{step} function gets given the \hs{k} to build, the dependencies as \hs{Maybe [k]} (where they are \hs{Nothing} if the rule is truely monadic), a function to demand the result of a dependent key, and a function to compute the current value and return the actual dependencies. To ensure each file is only visited once in a single execution a piece of temporary state is used with \hs{getTemp} and \hs{modifyTemp} to record which files were visited - this state is not persisted to the \hs{i} info parameter and is reset on each fresh execution of \hs{M}.

\subsection{\Bazel}\label{sec-implementation-bazel}

Now we have seen the three dependency schemes, we can directly reuse \hs{topological} to define Bazel. Furthermore, as Bazel is a tracing build system, we can reuse \hs{Trace} and \hs{traceMatch}, as part of the \hs{Traces} type. We define \Bazel as:

\begin{minted}{haskell}
bazel :: Hashable v => Build Applicative (Traces k v) k v
bazel = topological $ \k dk act -> do
    poss <- traceMatch (\k v -> (== v) <$> getStoreHash k) k . traces =<< getInfo
    if null poss then do
        v <- act
        dh <- mapM getStoreHash dk
        modifyInfo $ \i -> i
            {traces = Trace k (zip dk dh) (getHash v) : traces i
            ,contents = Map.insert (getHash v) v $ contents i}
        putStore k v
    else do
        h <- getStoreHash k
        when (h `notElem` poss) $
            putStore k . (Map.! head poss) . contents =<< getInfo
\end{minted}

We execute keys in a topological order. For each key we first get a list of the traces that match. If there are none we rebuild and record the trace, much like \Shake, but also recording the value of the result. If there are some matching traces, and the current value of \hs{k} is not compatible, we grab the first possiblity and grab its value from \hs{contents}.

\subsection{Cloud \Shake}\label{sec-implementation-cloud-shake}

Using the abstractions and approaches built thus far, we have shown how to combine dependency scheme and change approach to reproduce existing build systems. In the attached materials we have implemented 9 build systems corresponding to all three dependency schemes, matched with all three change approaches. To us, the most interesting build system as yet unavailable would matching recursive ordering with constructive traces - providing a cloud-capable build system with minimality, cut-off and monadic dependencies. Using our framework it is possible to define and test such a system:

\begin{minted}{haskell}
cloudShake :: Hashable v => Build Monad (Traces k v) k v
cloudShake = recursive $ \k _ ask act -> do
    poss <- traceMatch (\k v -> (== v) . getHash <$> ask k) k . traces =<< getInfo
    if null poss then do
        (dk, v) <- act
        dh <- mapM getStoreHash dk
        modifyInfo $ \i -> i
            {traces = Trace k (zip dk dh) (getHash v) : traces i
            ,contents = Map.insert (getHash v) v $ contents i}
        putStore k v
    else do
        now <- getStoreHash k
        when (now `notElem` poss) $
            putStore k . (Map.! head poss) . contents =<< getInfo
\end{minted}

The differences from \hs{bazel} are minor - the dependency scheme has changed from \hs{topological} to \hs{recursive}, and thus the dependency keys \hs{dk} are captured from the action rather than in advance, and the checking calls \hs{ask} to get the result instead of accessing the store directly. We believe this accurately captures and can be used to reason about how to make \Shake work with a cloud-based cache -- a challenge we intend to attempt in the future.

\subsection{Smarter \hs{[Trace]} Structures}\label{sec-smart-traces}

In the examples above, we have used \hs{[Trace]} to capture a list of traces stored in the info field, which accurately models the data and operations. However, using a list necessarily means that finding the right trace takes $O(n)$. For each of the \hs{Trace} based systems it is possible to devise a smarter representation, which we sketch below. Note that these implementations do not avoid calls to \hs{compute}, merely overheads in the build system itself.

\begin{enumerate}
\item Any system using verifying traces, e.g. \Shake, is unlikely to see significant benefit from storing more than one \hs{Trace} per key\footnote{There is a small chance of a benefit if the dependencies change but the result does not, and then the dependencies change back to what they were before.}. Therefore, such systems can store \hs{Map k (Trace k v)}, where the initial \hs{k} is the \hs{key} field of \hs{Trace}.
\item Any system using \hs{Applicative} dependencies can omit the dependency keys from the \hs{Trace} as they can be recovered from the \hs{key} field.
\item Any \hs{Applicative} system storing constructive traces, e.g. \Bazel, can index directly from the key and results to the output result - i.e. \hs{Map (k, [Hash v]) (Hash v)}. Importantly, assuming the traces are stored on a central server, the client can compute the key and the hashes of its dependencies, then make a single call to the server to retrieve the result hash. In this formulation we have removed the possibilty for a single key/dependency state to map to multiple different hashes, e.g. on a non-determinstic build - something \Bazel already prohibits which is discussed more in \S\ref{sec-non-determinism}.
\item Finally, a \hs{Monad} system with constructive traces can be efficiently encoded as \hs{Map k (Choice k v)}, assuming a definition of \hs{Choice} as:
\begin{minted}{haskell}
data Choice k v = Choice k (Map (Hash v) Choice)
                | Result (Hash v)
\end{minted}
Here the \hs{Choice} effectively encodes a tree, asking successive questions about keys, and taking different branches based on the resulting value, until it reaches a final result. Implementing this structure over client-server communication requires either a chatty interface with lots of round-trips per \hs{Choice} step, or sending over a part of the tree that is not subsequently explored.
\end{enumerate}

As an aside we note that the optimal cloud structure for applicative and monadic constructive systems has a resemblance to the free dependency structures described in \S\ref{sec-free-build}.

