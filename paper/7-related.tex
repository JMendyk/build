\section{Related work}\label{sec-related}

While there is research on individual build systems, there has been little
research to date comparing different build systems. In~\S\ref{sec-background} we
covered several important build systems~--~in this section we relate a few
other build systems to our abstractions, and discuss other work where similar
abstractions~arise.

\subsection{Other Build Systems}\label{sec-related-build}

Most build systems, when viewed at the level we talk, can be captured with minor variations on the code presented in \S\ref{sec-implementations}. As some examples:

\begin{itemize}
\item \Ninja~\cite{ninja} combines the dependency strategy of \Make with the
validating traces of \Shake~--~our associated implementation provides such a
combination. Ninja is also capable of modelling rules that produce multiple
results, a limited form of polymorphism \S\ref{sec-polymorphism}.

\item \Tup~\cite{tup} functions much like \Make, but with a refined dirty-bit
implementation that watches the file system for changes and can thus avoid
rechecking the entire graph, and with automatic deleting of stale results.

\item \Redo~\cite{redo} almost exactly matches \Shake at the level of detail
given here, differing only on aspects like polymorphic
dependencies~\S\ref{sec-polymorphism}.

\item \Buck~\cite{buck} is very similar to \Bazel. \CloudBuild~\cite{esfahani2016cloudbuild}
differs by allowing non-determinism~\S\ref{sec-non-determinism}, thus more closely modelling
our original definition of \Bazel from \S\ref{sec-implementation-bazel}.

\item \Nix~\cite{dolstra2004nix} has coarse-grained dependencies, with precise
hashing of dependencies and downloading of precomputed build products. When
combined with \cmd{import-from-derivation}, \Nix can also be considered monadic,
making it similar to Cloud \Shake from~\S\ref{sec-implementation-cloud-shake}.
However, \Nix is not intended as a build system, and the coarse grained nature
(packages, not individual files) makes it targeted to a different purpose.
% John Ericson suggested in a blog comment that nix may be somewhat monadic, see:
% \url{https://blogs.ncl.ac.uk/andreymokhov/cloud-and-dynamic-builds/\#comment-1849}.

\item \Pluto~\cite{erdweg2015pluto} is based on a similar model to \Shake, but
additionally allows cyclic build rules combined with a user-specific resolution
strategy. Often such a strategy can be unfolded into the user rules without loss
of precision, but a fully general resolution handler extends the \hs{Task}
abstraction with additional features.
\end{itemize}

The one build system we are aware of that cannot be modelled in our framework is
\Fabricate~\cite{fabricate}. In \Fabricate a build system is a script which is
run in-order, in the spirit of\footnote{\Fabricate requires scripts to be
written in Python, but those details are not fundamental to what makes
\Fabricate special.}:

\begin{minted}[xleftmargin=10pt]{bash}
gcc -c util.c
gcc -c main.c
gcc util.o main.o -o main.exe
\end{minted}

\noindent
To achieve minimality, each separate command is traced at the OS-level, allowing
\Fabricate to record a trace entry stating that \cmd{gcc -c util.c} reads from
\cmd{util.c}. In future runs \Fabricate runs the script from start to finish,
skipping any commands where no inputs have changed.

Taking our abstraction, it is possible to encode \Fabricate assuming that
commands like \cmd{gcc -c util.c} are keys, there is a linear dependency between
each successive node, and that the OS-level tracing can be lifted back as a
monadic \hs{Task} function\footnote{In fact, \Shake has an execution mode that
can model \Fabricate{}-like build systems~--~see \hs{Development.Shake.Forward}
in the \Shake library.}. However, in our pure model the mapping is not perfect
as \cmd{gcc} writes to arbitrary files whose locations are not known in advance.

\todo{AM}{Write a short section about memoization if enough time.}
% \subsection{Memoization}\label{sec-related-memo}

\subsection{Self-adjusting computation}

While not typically considered build systems, self-adjusting computation is a
well studied area, and in particular the contrast between different formulations
has been well studied.

\todo{AM}{Write this section.}

\subsection{Profunctor Optics}\label{sec-related-optics}

The definition of \hs{Task} is:

\begin{minted}[xleftmargin=10pt]{haskell}
type Task c k v = @\std{forall}@ f. c f => (k -> f v) -> k -> Maybe (f v)
\end{minted}

\noindent
Which looks tantalisingly close to the profunctor optics definition by
\cite{pickering2017profunctor}:

\begin{minted}{haskell}
type Optic p a b s t = p a b -> p s t
\end{minted}

Provided we instantiate \hs{p} to something like
\hs{k}~\hs{->}~\hs{f}~\hs{v}~--~which many of the actual instances in that paper
do. The properties of such optics are well studied, and the functions like
\hs{dependencies} are very much based on observations from that field of work.
Alas, we have been unable to remove the \hs{Maybe} used to encode whether a file
is an input, without complicating other aspects of our definition. Furthermore,
the \hs{Build} operation lacks any further such symmetry.
