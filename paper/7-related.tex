\section{Related work}\label{sec-related}

While there is research on individual build systems, there has been little
research to date comparing different build systems. In~\S\ref{sec-background} we
covered several important build systems~--~in this section we relate a few
other build systems to our abstractions, and discuss other work where similar
abstractions~arise.

\subsection{Other Build Systems}\label{sec-related-build}

Most build systems, when viewed at the level we talk, can be captured with minor
variations on the code presented in \S\ref{sec-implementations}. Below we list
some notable examples:

\begin{itemize}
\item \Dune~\cite{dune} is a build system designed for OCaml/Reason projects.
Its distinguishing feature is that it uses
\emph{arrows}~\cite{hughes2000generalising} rather than monads to model
dynamic dependencies, which simplifies static dependency approximation.

\item \Ninja~\cite{ninja} combines the \hs{topological} scheduler of \Make with
the verifying traces of \Shake~--~our associated implementation provides such a
combination. \Ninja~is also capable of modelling build rules that produce
multiple results, a limited form of polymorphism \S\ref{sec-polymorphism}.

\item \Nix~\cite{dolstra2004nix} has coarse-grained dependencies, with precise
hashing of dependencies and downloading of precomputed build products. We
provided a model of \Nix in \S\ref{sec-implementation-cloud}, although it is
worth noting that \Nix is not primarily intended as a build system, and the
coarse grained nature (packages, not individual files) makes it targeted to a
different purpose.

\item \Pluto~\cite{erdweg2015pluto} is based on a similar model to \Shake, but
additionally allows cyclic build rules combined with a user-specific resolution
strategy. Often such a strategy can be unfolded into the user rules without loss
of precision, but a fully general resolution handler extends the \hs{Task}
abstraction with additional features.

\item \Redo~\cite{redo-idea}\cite{grosskurth2007redo}\cite{redo} almost exactly
matches \Shake at the level of detail given here, differing only on aspects like
polymorphic dependencies~\S\ref{sec-polymorphism}.

\item \Tup~\cite{tup} functions much like \Make, but with a refined dirty-bit
implementation that watches the file system for changes and can thus avoid
rechecking the entire graph. \Tup also automatically deletes stale results.
\end{itemize}

The one build system we are aware of that cannot be modelled in our framework is
\Fabricate by \citet{fabricate}. In \Fabricate a build system is a script that is
run in-order, in the spirit of:
% I'm not sure the footnote adds much, let's save space?
% \footnote{\Fabricate requires scripts to be
% written in Python, but those details are not fundamental to what makes
% \Fabricate special.}

\begin{minted}[xleftmargin=10pt]{bash}
gcc -c util.c
gcc -c main.c
gcc util.o main.o -o main.exe
\end{minted}

\noindent
To achieve minimality, each separate command is traced at the OS-level, allowing
\Fabricate to record a trace entry stating that \cmd{gcc -c util.c} reads from
\cmd{util.c}. In future runs \Fabricate runs the script from start to finish,
skipping any commands where no inputs have changed. The key difference from our
\hs{Tasks} abstraction is that instead of supplying a mapping from output to action,
\Fabricate supplies a list of statements, in an order, without declaring what each line produces.
There is no need to schedule the statements, and not enough information to do so.

Taking our abstraction, it is possible to encode \Fabricate assuming that
commands like \cmd{gcc -c util.c} are keys, there is a linear dependency between
each successive key, and that the OS-level tracing can be lifted back as a
monadic \hs{Task} function\footnote{\Shake provides support for
\Fabricate{}-like build systems~--~see \cmd{Development.Shake.Forward}
in the \Shake library.}. However, in our pure model the mapping is not perfect
as \cmd{gcc} writes to arbitrary files whose locations are not known in advance.

\subsection{Self-adjusting computation}

While not typically considered build systems, self-adjusting computation is a
well studied area, and in particular the contrast between different formulations
has been thoroughly investigated, e.g. see~\cite{acar2007selfadjusting}.
Self-adjusting computations can automatically adjust to an external change
to their inputs. A classic example is a self-adjusting sorting algorithm, which
can efficiently (in $O(\log{n})$ time where $n$ is the length of the input)
recalculate the result given an incremental change of the input. While very
close to build systems in spirit, self-adjusting computations are mostly used
for in-memory computation and rely on the ability to dynamically allocate new
keys in the store for sharing intermediate computations~--~an intriguing feature
rarely seen in build systems (\Shake's oracles~\S\ref{sec-polymorphism} can be
used to model this feature to a limited degree).

A lot of research has been dedicated to finding efficient data structures and
algorithms for self-adjusting computations (with a few open-source
implementations, e.g. \Incremental~\cite{incremental}). We plan to investigate
how these insights can be utilised by build systems as future work.

\subsection{Memoization}\label{sec-related-memo}

\emph{Memoization} is a classic optimisation technique for storing values of a
function instead of recomputing them each time the function is called. Minimal
build systems (see the Definition~\ref{def-minimal}) certainly perform
memoization: they \emph{store values instead of recomputing them each time}.
Memoization can therefore be reduced to a minimal build system (as we
demonstrate below), but not vice versa, since minimal build systems solve a more
complex optimisation problem.

As a simple example of using a build system for memoization, we solve a textbook
dynamic programming problem~--~Levenshtein's \emph{edit
distance}~\cite{levenshtein1966binary}: given two input strings $a$ and
$b$, find the shortest series of edit operations that transforms $a$
to $b$. The edit operations are typically \emph{inserting}, \emph{deleting} or
\emph{replacing} a symbol. The dynamic programming solution of this problem is
so widely known (e.g., see~\cite{cormen2001introduction}) that we provide its
encoding in our \hs{Tasks} abstraction without further explanation. We address
elements of strings $a_i$ and $b_i$ by keys \hs{A}~$i$ and \hs{B}~$i$,
respectively, while the cost of a subproblem $c_{ij}$ is identified by
\hs{C}~$i$~$j$.

\vspace{0.5mm}
\begin{minted}[xleftmargin=10pt, fontsize=\small]{haskell}
data Key = A Int | B Int | C Int Int deriving Eq
\end{minted}
\vspace{0mm}
\begin{minted}[xleftmargin=10pt, fontsize=\small]{haskell}
editDistance :: Tasks Monad Key Int
editDistance (C i 0) = Just $ Task $ const $ pure i
editDistance (C 0 j) = Just $ Task $ const $ pure j
editDistance (C i j) = Just $ Task $ \fetch -> do
    ai <- fetch (A i)
    bj <- fetch (B j)
    if ai == bj
        then fetch (C (i - 1) (j - 1))
        else do insert  <- fetch (C  i      (j - 1))
                delete  <- fetch (C (i - 1)  j     )
                replace <- fetch (C (i - 1) (j - 1))
                return (1 + minimum [insert, delete, replace])
editDistance _ = Nothing
\end{minted}
\vspace{0.5mm}

\noindent
When asked to build \hs{C}~$n$~$m$, a minimal build system will calculate the
result using memoization. Furthermore, when an input symbol $a_i$ is changed,
only necessary, incremental recomputation will be performed~--~an optimisation
that cannot be achieved just with memoization.

Self-adjusting computation, memoization and build systems are inherently related
topics, which poses the question of whether there is an underlying common
abstraction waiting to be discovered.

\begin{comment}
\subsection{Profunctor Optics}\label{sec-related-optics}

The definition of \hs{Task} is:

\begin{minted}[xleftmargin=10pt]{haskell}
newtype Task c k v = Task { run :: @\std{forall}@ f. c f => (k -> f v) -> f v }
\end{minted}

\noindent
Which looks tantalisingly close to the profunctor optics definition by
\cite{pickering2017profunctor}:

%TODO: Add a reference to the paper about forall f. c f => (k -> f v) -> f a
\begin{minted}[xleftmargin=10pt]{haskell}
type Optic p a b s t = p a b -> p s t
\end{minted}
\noindent
Provided we instantiate \hs{p}~\hs{a}~\hs{b} to something like
\hs{k}~\hs{->}~\hs{f}~\hs{v}~--~which many of the actual instances in that paper
do. The properties of such optics are well studied, and the functions like
\hs{dependencies} are very much based on observations from that field of work.
Alas, we have been unable to remove the \hs{Maybe} used to encode whether a file
is an input, without complicating other aspects of our definition. Furthermore,
the \hs{Build} abstraction lacks any further such symmetry.
\end{comment}
