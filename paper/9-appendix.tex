\clearpage
\section{Appendix}\label{sec-appendix}

\subsection{Compute transformers}\label{sec-appendix-transformers}

In this section we clarify some of the compute transformers used in this paper.

\hs{execute} uses the transformation based on the \hs{Identity} monad, feeding
\hs{fetch k = pure (store k)} to the compute:

\begin{minted}[xleftmargin=10pt]{haskell}
execute :: Compute Monad k v -> (k -> v) -> k -> Maybe v
execute compute store = fmap runIdentity . compute (pure . store)
\end{minted}
\vspace{1mm}
\begin{minted}[xleftmargin=10pt]{haskell}
newtype Identity a = Identity { runIdentity :: a }
\end{minted}
\vspace{1mm}
\begin{minted}[xleftmargin=10pt]{haskell}
instance Functor Identity where
    fmap f (Identity a) = Identity (f a)
\end{minted}
\vspace{1mm}
\begin{minted}[xleftmargin=10pt]{haskell}
instance Applicative Identity where
    pure a = Identity a
    Identity f <*> Identity a = Identity (f a)
\end{minted}
\vspace{1mm}
\begin{minted}[xleftmargin=10pt]{haskell}
instance Monad Identity where
    Identity a >>= f = f a
\end{minted}
\vspace{1mm}

\todo{AM}{Explain \hs{track}.}



Here is a draft implementation of \hs{inputs} used in the definition of
build system correctness in \S\ref{sec-build-correctness}:

\begin{minted}[xleftmargin=10pt]{haskell}
inputs :: Eq k => Task Monad k v -> Store i k v -> k -> [k]
inputs task store key = filter (isInput task) (closure deps key)
  where
    deps k = maybe [] snd (track task store k)

closure :: Eq a => (a -> [a]) -> a -> [a] -- Standard graph transitive closure

data Proxy a = Proxy

isInput :: Task Monad k v -> k -> Bool
isInput task = isNothing . task (const Proxy)
\end{minted}

\subsection{Compute examples}\label{sec-appendix-compute-examples}

\todo{AM}{Add some explanatory text.}

The \emph{Collatz sequence} $C_i$ is defined as follows:

\[
C_{i} = {\begin{cases}~n&{\text{for }}i=0\\~f(C_{i-1})&{\text{otherwise}},\end{cases}}\hspace{12pt}\text{where}\hspace{12pt}f(k)={\begin{cases}~k/2&{\text{if }}k\text{ is even}\\~3k+1&{\text{otherwise}}\end{cases}}
\vspace{2mm}
\]
\noindent
and $n$ is a positive integer parameter. The famous \emph{Collatz conjecture}
states that the Collatz sequence eventually reaches 1 for all possible values of
$n$. For example, if $n=6$, we reach 1 in eight steps:
$(6, 3, 10, 5, 16, 8, 4, 2, 1, \dots)$, after which the sequence loops forever:
$(4, 2, 1, 4, 2, 1, \dots)$.

We can express the computation of values in the Collatz sequence as a functorial
compute:

\begin{minted}[xleftmargin=10pt]{haskell}
data Collatz = Collatz Int

collatz :: Compute Functor Collatz Int
collatz get (Collatz k) | k <= 0    = Nothing
                        | otherwise = Just $ f <$> get (Collatz (k - 1))
  where
    f n | even n    = n `div` 2
        | otherwise = 3 * n + 1
\end{minted}

...

The \emph{generalised Fibonacci sequence} $F_i$ is defined as follows:

\[
F_{i} = {\begin{cases}~n&{\text{for }}i=0\\~m&{\text{for }}i=1\\~F_{i-1}+F_{i-2}&{\text{otherwise}}\end{cases}}
\vspace{2mm}
\]
\noindent
where $n$ and $m$ are integer parameters. By setting $n=0$ and $m=1$ we obtain
the famous \emph{Fibonacci sequence}: $(0, 1, 1, 2, 3, 5, 8, 13, \dots$), and if
$n=2$ and $m=1$, the result is the \emph{Lucas sequence}:
$(2, 1, 3, 4, 7, 11, 18, 29, \dots)$.

We can express the computation of values in the generalised Fibonacci sequence
as an applicative compute:

\begin{minted}[xleftmargin=10pt]{haskell}
data Fibonacci = Fibonacci Int

fibonacci :: Compute Applicative Fibonacci Int
fibonacci get (Fibonacci k) | k <= 1    = Nothing
                            | otherwise = Just $ (+) <$> get (Fibonacci (k - 1))
                                                     <*> get (Fibonacci (k - 2))
\end{minted}

...

The \emph{Ackermann function} $A(m, n)$ is defined as follows:

\[
A(m, n) = {\begin{cases}~n+1&{\text{for }}m=0\\~A(m-1, 1)&{\text{for }}n=0\\~A(m-1,A(m,n-1))&{\text{otherwise}}\end{cases}}
\vspace{2mm}
\]
\noindent
We can express the computation of the Ackermann function as a monadic compute:

\begin{minted}[xleftmargin=10pt]{haskell}
data Ackermann = Ackermann Int Int

ackermann :: Compute Monad Ackermann Int
ackermann get (Ackermann m n)
    | m < 0 || n < 0 = Nothing
    | m == 0    = Just $ return (n + 1)
    | n == 0    = Just $ get (Ackermann (m - 1) 1)
    | otherwise = Just $ do
        index <- get (Ackermann m (n - 1))
        get (Ackermann (m - 1) index)
\end{minted}