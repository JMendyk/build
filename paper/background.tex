\clearpage

\section{Background}\label{sec-background}

Below we define basic notions used in build systems and other similar domains,
for example, spreadsheets.

\subsection{Keys, values, hashes and store}

\emph{Keys} are used to distinguish \emph{values}. In build systems keys are
typically filenames, e.g. \textsf{src/file.c}, whereas values are file contents
(a C program source code in this case). In spreadsheets keys are cell names,
e.g. \textsf{A1}, and values are numbers, text, etc. that are typically displayed
inside cells. We will use type variables \hs{k} and \hs{v} to denote keys and
values, respectively.

A \emph{store} associates keys to values. It is convenient to assume that a store
is total, i.e. it contains a value for every possible key. We therefore also
assume that the type of values is capable of encoding values corresponding to
non-existent keys (missing files or empty cells).

We use a cryptographic \emph{hash function} \hs{hash :: v -> Hash v} for
efficient tracking and sharing of build results.

\subsection{Input, intermediate and output values}

Some values must be provided by the user as \emph{input}. For example,
\textsf{src/file.c} can be edited by the user who relies on the build system to
compile it into \textsf{obj/file.o}. Similarly, the user can input \textsf{A1 = 5}
and \textsf{B1 = 9} expecting the spreadsheet to compute their sum in \textsf{C1},
i.e. \textsf{C1 = 14}.

In the above examples, \textsf{obj/file.o} and \textsf{C1} are \emph{output} values.

In some situations we might also need the notion of \emph{intermediate} values,
which are not interesting for the user but are produced in the process of turning
inputs into outputs. For example, the user might only be interested in the
executable \textsf{bin/file.exe} obtained by linking \textsf{obj/file.o} with
standard libraries, in which case \textsf{obj/file.o} can be considered an
intermediate value.

\subsection{Non-deterministic computations}

Build systems and spreadsheets compute output values from input and intermediate
values. In the most typical case, these \emph{computations} are \emph{functions},
such as \textsf{C1 = A1 + B1}, i.e. their result is uniquely determined by the
input values. However, in general they can be \emph{relations}, i.e. have
multiple valid results. A spreadsheet example: \textsf{A2 = A1 + RANDOM(1,6)}.
This computation has six valid results for each input value \textsf{A1}. In
build systems, the object file \textsf{obj/file.o} is sometimes not uniquely
determined by the source \textsf{src/file.c} -- different compiler runs may
produce different valid results.

\subsection{Dynamic dependencies}

Many existing build systems need to know all the dependencies between files or
cells, i.e. the dependency graph, at the start of the build process. This makes
it possible to analyse the graph ahead of time, but is fundamentally limited:
parts of the dependency graph can often be discovered only during the build
process, i.e. the dependency graph is dynamic, not static. This is illustrated
by the following example.
\todo{AM}{Add two examples: include files and cyclic spreadsheet computations.}

\subsection{Requirements for build systems}

\begin{itemize}
    \item Correctness
    \item Minimality
    \item Support for sharing and skipping intermediate values
\end{itemize}
...
