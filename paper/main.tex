%% For double-blind review submission, w/o CCS and ACM Reference (max submission space)
\documentclass[acmsmall,review,anonymous]{acmart}\settopmatter{printfolios=true,printccs=false,printacmref=false}
%% For double-blind review submission, w/ CCS and ACM Reference
%\documentclass[acmsmall,review,anonymous]{acmart}\settopmatter{printfolios=true}
%% For single-blind review submission, w/o CCS and ACM Reference (max submission space)
%\documentclass[acmsmall,review]{acmart}\settopmatter{printfolios=true,printccs=false,printacmref=false}
%% For single-blind review submission, w/ CCS and ACM Reference
%\documentclass[acmsmall,review]{acmart}\settopmatter{printfolios=true}
%% For final camera-ready submission, w/ required CCS and ACM Reference
%\documentclass[acmsmall]{acmart}\settopmatter{}

%% Journal information
%% Supplied to authors by publisher for camera-ready submission;
%% use defaults for review submission.
\acmJournal{PACMPL}
\acmVolume{1}
\acmNumber{ICFP} % CONF = POPL or ICFP or OOPSLA
\acmArticle{1}
\acmYear{2018}
\acmMonth{1}
\acmDOI{}
\startPage{1}

%% Copyright information
%% Supplied to authors (based on authors' rights management selection;
%% see authors.acm.org) by publisher for camera-ready submission;
%% use 'none' for review submission.
\setcopyright{none}
%\setcopyright{acmcopyright}
%\setcopyright{acmlicensed}
%\setcopyright{rightsretained}
%\copyrightyear{2018}           %% If different from \acmYear

\bibliographystyle{ACM-Reference-Format}
%% Note: author/year citations are required for papers published as an
%% issue of PACMPL.
\citestyle{acmauthoryear}

%%%%%%%%%%%%%%%%%%%%%%%%%%%%%%%%%%%%%%%%%%%%%%%%%%%%%%%%%%%%%%%%%%%%%%
%% Note: Authors migrating a paper from PACMPL format to traditional
%% SIGPLAN proceedings format must update the '\documentclass' and
%% topmatter commands above; see 'acmart-sigplanproc-template.tex'.
%%%%%%%%%%%%%%%%%%%%%%%%%%%%%%%%%%%%%%%%%%%%%%%%%%%%%%%%%%%%%%%%%%%%%%

\usepackage{bookmark}
\usepackage{booktabs}
\usepackage{subcaption}
\usepackage[utf8]{inputenc}
\usepackage[T1]{fontenc}

% Haskell code snippets and useful shortcuts
\usepackage{minted}
\setminted[haskell]{escapeinside=@@}
\newcommand{\hs}{\mintinline{haskell}}
\newcommand{\teq}{\smaller $\sim$}
\newcommand{\ghci}{$\lambda$>}
\newcommand{\defeq}{\stackrel{\text{def}}{=}}
\newcommand{\std}[1]{{\color[rgb]{0,0.3,0} #1}}
\newcommand{\blk}[1]{{\color[rgb]{0,0,0} #1}}

\begin{document}

%% Title information
\title[Build systems]{Build systems \`a la Carte}
% \titlenote{with title note}
% \subtitle{Subtitle}
% \subtitlenote{with subtitle note}

%% Author information
%% Contents and number of authors suppressed with 'anonymous'.
%% Each author should be introduced by \author, followed by
%% \authornote (optional), \orcid (optional), \affiliation, and
%% \email.
%% An author may have multiple affiliations and/or emails; repeat the
%% appropriate command.
%% Many elements are not rendered, but should be provided for metadata
%% extraction tools.

%% Author with single affiliation.
\author{First1 Last1}
\authornote{with author1 note}          %% \authornote is optional;
                                        %% can be repeated if necessary
\orcid{nnnn-nnnn-nnnn-nnnn}             %% \orcid is optional
\affiliation{
  \position{Position1}
  \department{Department1}              %% \department is recommended
  \institution{Institution1}            %% \institution is required
  \streetaddress{Street1 Address1}
  \city{City1}
  \state{State1}
  \postcode{Post-Code1}
  \country{Country1}                    %% \country is recommended
}
\email{first1.last1@inst1.edu}          %% \email is recommended

%% Author with two affiliations and emails.
\author{First2 Last2}
\authornote{with author2 note}          %% \authornote is optional;
                                        %% can be repeated if necessary
\orcid{nnnn-nnnn-nnnn-nnnn}             %% \orcid is optional
\affiliation{
  \position{Position2a}
  \department{Department2a}             %% \department is recommended
  \institution{Institution2a}           %% \institution is required
  \streetaddress{Street2a Address2a}
  \city{City2a}
  \state{State2a}
  \postcode{Post-Code2a}
  \country{Country2a}                   %% \country is recommended
}
\email{first2.last2@inst2a.com}         %% \email is recommended
\affiliation{
  \position{Position2b}
  \department{Department2b}             %% \department is recommended
  \institution{Institution2b}           %% \institution is required
  \streetaddress{Street3b Address2b}
  \city{City2b}
  \state{State2b}
  \postcode{Post-Code2b}
  \country{Country2b}                   %% \country is recommended
}
\email{first2.last2@inst2b.org}         %% \email is recommended

\begin{abstract}
Build systems are awesome. Build systems are terrifying. In this paper ...
\end{abstract}

%% 2012 ACM Computing Classification System (CSS) concepts
%% Generate at 'http://dl.acm.org/ccs/ccs.cfm'.
\begin{CCSXML}
<ccs2012>
<concept>
<concept_id>10011007.10011006.10011008</concept_id>
<concept_desc>Software and its engineering~General programming languages</concept_desc>
<concept_significance>500</concept_significance>
</concept>
<concept>
<concept_id>10003456.10003457.10003521.10003525</concept_id>
<concept_desc>Social and professional topics~History of programming languages</concept_desc>
<concept_significance>300</concept_significance>
</concept>
</ccs2012>
\end{CCSXML}

\ccsdesc[500]{Software and its engineering~General programming languages}
\ccsdesc[300]{Social and professional topics~History of programming languages}
%% End of generated code

% \keywords{functional programming, build systems}

\maketitle

\section{Introduction}\label{sec-intro}

Build systems (such as \Make) are big, complicated, but unloved part of
the software ecosystem.  Every developer on the planet uses one, but
they are very much a means to an end, and seldom the focus of
attention.  But the challenges of scale have driven large software firms
like Microsoft, Facebook, and Google to develop their own build
systems, each with its own choices and idiosyncrasies.

Seldom do people ask questions like ``What does it mean for my build
system to be correct?'' or ``What are the trade-offs between different
approaches?''.  Complex build systems use subtle algorithms, but they
are often hidden away, and not the object of study.

In this paper we offer a general framework in which to discuss these
questions in a way that is both abstract (omitting incidental detail)
and yet precise (implemented as Haskell code).  Specifically we make
these contributions:
\begin{itemize}
\item We describe some simple but novel abstractions that
crisply encapsulate what a build system is.
\item We show that we can instantiate
these abstractions to describe the essence of a variety of different
build systems, including \Make, \Shake, \Bazel, and \Excel, each in
a dozen lines of code or so.
Doing this in a single setting allows
the differences and similarities between these huge systems to be
brought out clearly.
\item Build systems vary on many axes;
for example: static vs dynamic dependencies; cloud-build, including
shallow vs deep; deterministic vs non-deterministic build rules;
early cut-off; self-tracking build systems; and persistent metadata.
These properties (which we define in~\S\ref{sec-background}) are often
deeply-built-in assumptions of a particular build system.
In contrast, our framework allows them to be distinguished,
reasoned about, and varied.
\item Two particularly desirable properties are dynamic dependencies
and cloud-build; yet no currently-available build system supports
both.  Our framework makes it rather easy to do so, for the first
time.
\end{itemize}
Thus equipped, instead of seeing build systems as unrelated
points in space, we can re-envisage them as locations in a landscape,
leading to better understanding of what they do and how they compare,
and suggesting exploration of other (as yet unoccupied points) in the
landscape.

Papers about "frameworks" are often fuzzy.  This one is not: all our
abstractions are defined in Haskell, and we have (freely-available)
executable models of all the build systems we describe.  An unusual
feature is that we include \Excel in our line-up because, looked at
in the right way, it certainly is a build system.

\todo{AM/NM}{Add a paragraph on the decomposition of build systems into
\emph{compute} and \emph{build} components, with the latter split into
\emph{when} and \emph{how} parts.}

\section{Build tasks, abstractly}\label{sec-compute}

This section presents purely functional abstractions for build systems that
allow us to express all the intricacies of build systems discussed in the
previous section~\S\ref{sec-background} and design complex build systems from
simple primitives. Specifically, we present the \emph{compute} and \emph{build}
abstractions:

\begin{minted}[xleftmargin=10pt]{haskell}
type Compute c k v = @\std{forall}@ f. c f => (k -> f v) -> k -> Maybe (f v)
type Build c i k v = Compute c k v -> k -> Maybe i -> (k -> v) -> (i, k -> v)
\end{minted}

\noindent
Compute represents build rules (in build systems) or formulas (in spreadsheets),
and is completely isolated from the world of compilers, file systems, memories,
caches, and all other complexities of real build systems. We explain the compute
type in~\S\ref{sec-general-compute}, and then discuss several noteworthy special cases:
functorial (\S\ref{sec-compute-functor}), applicative
(\S\ref{sec-compute-applicative}) and monadic (\S\ref{sec-compute-monad}) types
of compute.

Build corresponds to the algorithm used for bringing a key-value store to a
consistent state by recomputing dirty values in an appropriate order. We explain
the build type in~\S\ref{sec-build}, and then provide implementations for several
build systems in~\S\ref{sec-implementations}.

\subsection{Common vocabulary for build systems}
\label{sec-background-vocabulary}

In this section we introduce a common vocabulary for build systems that allows
us to abstract away from specific application domains, such as software
development or spreadsheets. We will use this vocabulary throughout the rest of
the paper.

\emph{Keys} are used to distinguish \emph{values}. In software build systems
keys are typically filenames, e.g. \cmd{main.c}, whereas values are file
contents (a C program source code in this case). In spreadsheets keys are cell
names, e.g. \cmd{A1}, and values are numbers, text, etc. that are typically
displayed inside cells. In Haskell code, we will use type variables \hs{k}
and \hs{v} to denote keys and values, respectively.

We use a cryptographic \emph{hash function} \hs{hash :: v -> Hash} for efficient
tracking and sharing of build results, where \hs{Hash} is an abstract data type
equipped with the equality test.

A \emph{store} associates keys to values. It is convenient to assume that a store
is total, i.e. it contains a value for every possible key. We therefore also
assume that the type of values is capable of encoding values corresponding to
non-existent keys (missing files or empty cells). This, in particular, suggests
that it is possible to \emph{depend on the absence of a value} which is a useful
feature for build systems. In addition to usual \hs{get} and \hs{put} operations,
some stores support the \hs{getHash} operation implemented more efficiently than
by hashing the result of \hs{get}. For example, GVFS (Git Virtual File
System)~\cite{gvfs} stores file hashes and downloads file contents only on demand.

Some values must be provided by the user as \emph{input}. For example,
\cmd{main.c} can be edited by the user who relies on the build system to
compile it into \cmd{main.o} and subsequently \cmd{main.exe}. End build products,
such as \cmd{main.exe}, are \emph{output} values. All other values (in this case
\cmd{main.o}) are \emph{intermediate}; they are not interesting for the user
but are produced in the process of turning inputs into outputs.

\subsection{Compute}\label{sec-general-compute}

...
% \begin{figure}
% \begin{minted}{haskell}
% type Compute c k v = @\std{forall}@ f. c f => (k -> f v) -> k -> Maybe (f v)
% \end{minted}
% \caption{Compute abstractions}\label{fig-compute}
% \end{figure}
% Consider abstractions in Fig.~\ref{fig-compute}.

\subsection{Functorial compute}\label{sec-compute-functor}

The \emph{Collatz sequence} $C_i$ is defined as follows:

\[
C_{i} = {\begin{cases}~n&{\text{for }}i=0\\~f(C_{i-1})&{\text{otherwise}},\end{cases}}\hspace{12pt}\text{where}\hspace{12pt}f(k)={\begin{cases}~k/2&{\text{if }}k\text{ is even}\\~3k+1&{\text{otherwise}}\end{cases}}
\vspace{2mm}
\]
\noindent
and $n$ is a positive integer parameter. The famous \emph{Collatz conjecture}
states that the Collatz sequence eventually reaches 1 for all possible values of
$n$. For example, if $n=6$, we reach 1 in eight steps:
$(6, 3, 10, 5, 16, 8, 4, 2, 1, \dots)$, after which the sequence loops forever:
$(4, 2, 1, 4, 2, 1, \dots)$.

We can express the computation of values in the Collatz sequence as a functorial
compute:

\begin{minted}[xleftmargin=10pt]{haskell}
data Collatz = Collatz Int

collatz :: Compute Functor Collatz Int
collatz get (Collatz k) | k <= 0    = Nothing
                        | otherwise = Just $ f <$> get (Collatz (k - 1))
  where
    f n | even n    = n `div` 2
        | otherwise = 3 * n + 1
\end{minted}

...


\subsection{Applicative compute}\label{sec-compute-applicative}
The \emph{generalised Fibonacci sequence} $F_i$ is defined as follows:

\[
F_{i} = {\begin{cases}~n&{\text{for }}i=0\\~m&{\text{for }}i=1\\~F_{i-1}+F_{i-2}&{\text{otherwise}}\end{cases}}
\vspace{2mm}
\]
\noindent
where $n$ and $m$ are integer parameters. By setting $n=0$ and $m=1$ we obtain
the famous \emph{Fibonacci sequence}: $(0, 1, 1, 2, 3, 5, 8, 13, \dots$), and if
$n=2$ and $m=1$, the result is the \emph{Lucas sequence}:
$(2, 1, 3, 4, 7, 11, 18, 29, \dots)$.

We can express the computation of values in the generalised Fibonacci sequence
as an applicative compute:

\begin{minted}[xleftmargin=10pt]{haskell}
data Fibonacci = Fibonacci Int

fibonacci :: Compute Applicative Fibonacci Int
fibonacci get (Fibonacci k) | k <= 1    = Nothing
                            | otherwise = Just $ (+) <$> get (Fibonacci (k - 1))
                                                     <*> get (Fibonacci (k - 2))
\end{minted}

...

\subsection{Monadic compute}\label{sec-compute-monad}

The \emph{Ackermann function} $A(m, n)$ is defined as follows:

\[
A(m, n) = {\begin{cases}~n+1&{\text{for }}m=0\\~A(m-1, 1)&{\text{for }}n=0\\~A(m-1,A(m,n-1))&{\text{otherwise}}\end{cases}}
\vspace{2mm}
\]
\noindent
We can express the computation of the Ackermann function as a monadic compute:

\begin{minted}[xleftmargin=10pt]{haskell}
data Ackermann = Ackermann Int Int

ackermann :: Compute Monad Ackermann Int
ackermann get (Ackermann m n)
    | m < 0 || n < 0 = Nothing
    | m == 0    = Just $ return (n + 1)
    | n == 0    = Just $ get (Ackermann (m - 1) 1)
    | otherwise = Just $ do
        index <- get (Ackermann m (n - 1))
        get (Ackermann (m - 1) index)
\end{minted}

...

\subsection{Build}\label{sec-general-build}

...


%% Acknowledgments
\begin{acks}
  %% acks environment is optional
  %% contents suppressed with 'anonymous'
  %% Commands \grantsponsor{<sponsorID>}{<name>}{<url>} and
  %% \grantnum[<url>]{<sponsorID>}{<number>} should be used to
  %% acknowledge financial support and will be used by metadata
  %% extraction tools.
  We would like to thank ...
\end{acks}

% \bibliography{refs}

%% Appendix
% \appendix
% \section{Appendix}
% Text of appendix \ldots

\end{document}
