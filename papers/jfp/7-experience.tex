\section{Experience}\label{sec-experience}

As a set of authors, we've written a paper about the \Shake build system \cite{mitchell2012shake},
applying that build system to GHC \cite{hadrian}, and a paper about the structure of build
systems \cite{mokhov2018buildsystems}. The original purpose of that final paper was to create
``Cloud Shake'', and use it for the GHC build system. In this section we tie these papers together,
reflecting on our experience after all these papers.

\subsection{Experience from \Shake}

The original design of \Shake hasn't changed since the initial paper. Since that paper there have been roughly 5,000 commits to the \Shake project\footnote{\url{https://github.com/ndmitchell/shake}}. They add concepts like resources (if two rules content from a single external resource), rewrite serialisation to be faster, and documentation including a website, and add lots of tests. The biggest change in that time period was moving from blocking threads to continuations for the suspending scheduler. The most visible change is switching \hs{*>} for \hs{\%>}\footnote{A conflicting \hs{*>} operator was added to the Haskell \hs{Prelude}.}, but almost all external and internal details remain the same.

We consider the lack of change to suggest that \Shake is based on underlying principles -- principles we can now name and describe as a consequence of \cite{mokhov2018buildsystems}.

\subsection{Experience from building GHC with \Shake}

\todo{AM: Can you flesh this out}

The build system used to compile GHC, known as Hadrian, has continued development. The use of dynamic dependencies has made the build much easier, and it's now more maintainable. It's now merged into the GHC repo and tested as standard by the CI. It's current state is ... In particular, it takes N minutes to build with \Shake, vs M minutes to build with \Make. The \Shake version has attracted more contributors, and been easier to modify, etc. The code is N lines.

The design we originally outlined remains the one in use. There has been a huge amount of engineering work to make the tests pass (many of the details encoded in the \Make build system were incidental -- but some were essential).

One of the benefits of using Shake is that we have access to high quality profiling information, allowing us to compute critical paths and other metrics (see \cite{mitchell2019ghcrebuildtimes} for a tour of the metrics). This information has shown us that GHC is slow to build (40m on an 8 CPU machine on Windows), that more CPUs would not help (on unlimited CPUs it would take at least 37m), and that a handful of steps (two Haskell compiles, some calls to \hs{configure}) are responsible for a significant amount of that time (at least 10m).

\subsection{Experience from Cloud \Shake}\label{sec-cloud-shake}

Converting \Shake into Cloud \Shake wasn't a difficult process armed with a roadmap. The key was the introduction of two new functions:

\begin{minted}[xleftmargin=10pt]{haskell}
addCloud :: k -> Ver -> Ver -> [[(k, Hash v)]] -> v -> [FilePath] -> IO ()
lookupCloud :: (k -> m (Maybe (Hash v))) -> k -> Ver -> Ver -> m (Maybe (v, [[k]], IO ()))
\end{minted}

These functions are suspiciously like \hs{recordCT} and \hs{constructCT} from \S\ref{sec-constructive-traces}, with their differences perhaps the most illustrative of the changes required. (We have made the replacements, like replacing \hs{Key} for \hs{k}, to reduce irrelevant differences.)

\begin{itemize}
\item There are two \hs{Ver} arguments being passed to each operation. These are the versions of the build script, and the rule for this particular key. If either version changes then it is as though the key has changed, and nothing will match. These versions are important to avoid using stale build products from previous versions of the build script.
\item The list of dependencies to \hs{addCloud} is a list of lists, rather than a simple list. The reason is that \Shake allows a list of dependencies to be supplied at once, so they can all be built in parallel.
\item The \hs{addCloud} function also takes a list of \hs{FilePath}, being the files that this rule produces -- which must be declared with \hs{produces}, or the output keys from a rule.
\item The \hs{lookupCloud} allows an explicit \hs{Nothing} when looking up a dependent key, since some keys are not buildable.
\item The \hs{lookupCloud} returns at most one result, rather than a list. This change was made for simplicity.
\end{itemize}

The integration of these functions into \Shake is also interesting. We found the most expedient design was to leave \Shake with a verifying trace, but if the verifying trace doesn't work, we use then consult the constructive trace. By bolting constructive traces onto the side of Shake we avoid reengineering of the central database. We haven't found any significant downsides from the bolt-on approach thus far, so it may be a sensible route to go even if developing from scratch -- allowing an optimised verified trace implementation in many cases, and falling back to a more complex implementation only rarely.

The one thing we haven't yet completed on the engineering side is a move to hosting caches over HTTP. At the moment all caches are on shared file systems. This approach can use mounted drives to mirror HTTP connections onto file systems, and reuse tools for managing file systems, share caches with rsync, and is simple. Unfortunately, on certain OS's (e.g. Windows) mounting an HTTP endpoint as a file system requires administrator privileges, so an HTTP cache is still required.

\subsection{Experience from building GHC with Cloud \Shake}

\begin{itemize}
\item Absolute paths and system binaries
\item Deferred materialisation. Different invariants for (a) local builds, (b) cloud sharing (you must list all the things you produce), (c) sharing + deferred materialisation (you must declare all the things you consume). Interaction with early cut-off.
\item  Cloud stuff needed tracing infrastructure to expose dependencies.
\end{itemize}


Can reduce times a lot.

Needs to get much better dependencies.

Also been deployed by Standard Chartered.
