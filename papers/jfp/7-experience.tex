\section{Experience}\label{sec-experience}

We have presented a framework that can describe, and indeed execute in
prototype form, a wide spectrum of build systems.  But our ultimate
goal is a practical one: to use these insights to develop better
build systems. Our earlier work on \Shake~\cite{mitchell2012shake}, and applying
\Shake to building GHC~\cite{hadrian}, makes progress in that direction.

Based on the theory developed in this paper we have extended \Shake to
become \Cloud \Shake, the first cloud-capable build system to support
both early cutoff and monadic dependencies~(\S\ref{sec-implementation-cloud}),
and used it to implement GHC's (very substantial) build system,
\Hadrian~\cite{hadrian}. In this section we reflect on our experience of turning
theory into practice.

\vspace{-1mm}
\subsection{Haskell as a Design Language}\label{sec-design-lang}

Build systems are surprisingly tricky.  It is easy to waffle, and
remarkably hard to be precise. As this paper exemplifies, it is
possible to use Haskell as a \emph{design language}, to express quite
abstract ideas in a precise form -- indeed, precise enough to be executed.

Moreover, doing so is extremely beneficial.
The discipline of writing executable prototypes in
Haskell had a profound effect on our thinking. It forced misconceptions
to surface early.  It required us to be explicit about side effects.
It gave us a huge incentive to design abstractions that had simple types
and an explicable purpose.

Consider, for example, our \hs{Task} type:
\begin{minted}[xleftmargin=10pt]{haskell}
newtype Task c k v = Task (@\std{forall}@ f. c f => (k -> f v) -> f v)
\end{minted}
We started off with a much more concrete type, and explored
multiple variants.  During those iterations, this single line of
code gave us a tangible, concrete basis for productive conversations,
much more so than general conversations about ``tasks''.

It is also worth noting that we needed a rather expressive language,
to faithfully express the abstractions that seem natural in this
setting.  In the case of \hs{Task} we use a data constructor with a
polymorphic field; a higher-kinded type variable \hs{f}~\hs{::}~\hs{*}~\hs{->}~\hs{*}; an
even more abstracted kind for \hs{c}~\hs{::}~\hs{(*}~\hs{->}~\hs{*)}~\hs{->}~\hs{Constraint}; and,
of course, type classes. Our models have been translated to
Rust~\cite{translation_rust} and Kotlin~\cite{translation_kotlin}, and in both
cases there was a loss of precision due to language-specific limitations.

When thinking about the type constructors over which \hs{f} might usefully
range, it turned out that we could adopt the \emph{existing abstractions}
of \hs{Functor}, \hs{Applicative}, \hs{Monad} and so on.
That in turn led us to think in a new way about the taxonomy
of build systems -- see~\S\ref{sec-why-polymorphism}. In the other direction,
trying to express an \emph{existing build system} \Dune in our models led us to
finding a new abstraction -- the \hs{Selective} \mbox{type
class~\cite{mokhov_selective_2019}}, which turned out to be useful outside the
build systems domain.

The effect of using a concrete design language went well beyond
merely \emph{expressing} our ideas: it \emph{directly influenced}
our thinking.  For example, here are definitions for a scheduler and rebuilder,
from~\S\ref{sec-concrete-impl}:

\vspace{0.5mm}
\begin{minted}[fontsize=\small,xleftmargin=5pt]{haskell}
type Scheduler c i ir k v = Rebuilder c ir k v -> Build c i k v
type Rebuilder c   ir k v = k -> v -> Task c k v -> Task@\,@(MonadState ir) k v
\end{minted}
\vspace{0.5mm}

These powerful and modular abstractions, which ultimately formed part
of the conceptual structure of the paper, emerged fairly late in the
project, as we repeatedly reviewed, redesigned, and refactored our
executable prototypes. It is hard to believe that we could have
developed them without the support of Haskell as a design language.

\subsection{Experience from \Shake}\label{sec-experience-shake}

The original design of \Shake has not changed since the initial paper, but the
implementation has continued to mature -- there have been roughly 5,000
subsequent commits to the \Shake
project\footnote{See \url{https://github.com/ndmitchell/shake}.}. These commits
added concepts like \emph{resources} (for handling situations when two build tasks
contend on a single external resource), rewriting serialisation to be faster,
documentation including a website\footnote{See \url{https://shakebuild.com}.},
and add a lot of tests. The biggest change in that time period was an
implementation change: moving from
blocking threads to continuations for the suspending scheduler.
But otherwise, almost all external and internal details remain the
  same\footnote{The most visible change is purely notational: switching from
  \hs{*>} to \hs{@\%@>} for defining rules, because a conflicting \hs{*>}
  operator was added to the Haskell \hs{Prelude}.}.
We consider the lack of change suggestive that \Shake is based on fundamental
principles -- principles we can now name and describe as a consequence
of this paper.

There are two main
aspects to the original \Shake paper~\cite{mitchell2012shake} that are described
more clearly in this paper. Firstly, the rebuilder can now be described using
verifying step traces~\S\ref{sec-step-traces}, with a much clearer relationship
to the unoptimised verifying traces of~\S\ref{sec-verifying-traces}. Secondly,
in the original paper the tasks (there called ``actions'')
were described in continuation-passing style using the following data
type\footnote{The original paper uses concrete types \hs{Key} and \hs{Value}.
Here we generalise these types to~\hs{k} and~\hs{v}, and also add~\hs{a} so that
\hs{Action}~\hs{k}~\hs{v} can be an instance of \hs{Monad}.}:
\vspace{0.5mm}
\begin{minted}[xleftmargin=10pt]{haskell}
data Action k v a = Finished a
                  | Depends k (v -> Action k v a)
\end{minted}
\vspace{0.5mm}

\noindent
In this paper we describe tasks more directly, in a monadic (or applicative or functorial)
style. But in fact the two are equivalent: \hs{Task}~\hs{Monad}~\hs{k}~\hs{v} is
isomorphic to \hs{Action}~\hs{k}~\hs{v}~\hs{v}.
To be concrete, the functions \hs{toAction} and \hs{fromAction} defined below
witness the isomorphism in both directions.

\vspace{0.5mm}
\begin{minted}[xleftmargin=10pt]{haskell}
instance Monad (Action k v) where
    return = Finished
    Finished x    >>= f = f x
    Depends ds op >>= f = Depends ds $ \v -> op v >>= f
\end{minted}
\begin{minted}[xleftmargin=10pt]{haskell}
toAction :: Task Monad k v -> Action k v v
toAction (Task run) = run (\k -> Depends k Finished)
\end{minted}
\begin{minted}[xleftmargin=10pt]{haskell}
fromAction :: Action k v v -> Task Monad k v
fromAction x = Task (\fetch -> f fetch x)
  where
    f _     (Finished v  ) = return v
    f fetch (Depends d op) = fetch d >>= f fetch . op
\end{minted}
\vspace{0.5mm}

\noindent
Similarly, in the original paper \Make tasks were described as:

\vspace{1mm}
\begin{minted}[xleftmargin=10pt]{haskell}
data Rule k v a = Rule { depends :: [k], action :: [v] -> a }
\end{minted}
\vspace{1mm}
\noindent
Assuming the lengths of the lists \hs{[@@k]} and \hs{[@@v]} always match, the
data type \hs{Rule}~\hs{k}~\hs{v}~\hs{v} is isomorphic to
\hs{Task}~\hs{Applicative}~\hs{k}~\hs{v}, and we can define a similar
\hs{Applicative} instance and conversion functions.

By expressing these types
using \hs{Task} we are able to describe the differences more concisely
(\hs{Monad} vs \hs{Applicative}), use existing literature to determine what is
and isn't possible, and explore other constraints beyond just \hs{Monad} and
\hs{Applicative}.
These and other isomorphisms for \emph{second-order functionals}, i.e.
functions of the form

\vspace{1mm}
\begin{center}
\begin{minipage}{0.5\textwidth}
\begin{minted}[xleftmargin=10pt]{haskell}
@\std{forall}@ f. c f => (k -> f v) -> f a
\end{minted}
\end{minipage}
\end{center}
\vspace{1mm}

\noindent for various choices of~\hs{c}, are studied in depth by~Jaskelioff and
O'Connor~\shortcite{jaskelioff2015representation}.

\subsection{Experience from \Cloud \Shake}\label{sec-cloud-shake}

Converting \Shake into \Cloud \Shake was not a difficult process once armed with
the roadmap in this paper. The key was the introduction of two new
functions:

\vspace{1mm}
\begin{minted}[xleftmargin=10pt]{haskell}
addCloud :: k -> Ver -> Ver -> [[(k, Hash v)]] -> v -> [k] -> IO ()
lookupCloud :: (k -> m (Maybe (Hash v))) -> k -> Ver -> Ver
            -> m (Maybe (v, [[k]], IO ()))
\end{minted}
\vspace{1mm}

\noindent
These functions are suspiciously like \hs{recordCT} and \hs{constructCT}
from~\S\ref{sec-constructive-traces}, with their differences perhaps the most
illustrative of the changes required\footnote{We have made some minor changes from
actual \Shake, like replacing \hs{Key} for \hs{k}, to reduce irrelevant
differences.}.

\begin{itemize}
\item Two \hs{Ver} arguments are passed to each function. These are
      the versions of the build script, and the rule for this particular key. If
      either version changes then it is as though the key has changed, and
      nothing will match. These versions are important to avoid using stale
      build products from previous versions of the build script.
\item The list of dependencies to \hs{addCloud} is a list of lists, rather than
      a simple list. The reason is that \Shake allows a list of dependencies to
      be specified simultaneously, so they can all be built in parallel.
\item The \hs{addCloud} function also takes a list of keys \hs{[k]}, being the
      files that this rule produces. These produced files include those which
      are output keys from a rule and those declared with the function
      \hs{produces}.
\item The \hs{lookupCloud} function allows an explicit \hs{Nothing} when looking
      up a dependent key, since some keys are not buildable.
\item The \hs{lookupCloud} function returns at most one result, rather than a
      list. This change was made for simplicity.
\end{itemize}

To integrate these functions into \Shake we found the most expedient route was
to leave \Shake with verifying traces, but if the verifying trace does not
match, we consult the constructive trace. By bolting constructive traces onto
the side of \Shake we avoid re-engineering of the central database. We have not
found any significant downsides from the bolt-on approach thus far, so it may be
a sensible route to go even if developing from scratch -- allowing an optimised
verified trace implementation in many cases, and falling back to a more complex
implementation (requiring consulting remote servers) only rarely.

The one thing we have not yet completed on the engineering side is a move to
hosting caches over HTTP. At the moment all caches are on shared file systems.
This approach can use mounted drives to mirror HTTP connections onto file
systems, and reuse tools for managing file systems, share caches with
\cmd{rsync}, and is simple. Unfortunately, on certain operating systems (e.g.
Windows) mounting an HTTP endpoint as a file system requires administrator
privileges, so an HTTP cache is still desirable.

\subsection{Experience from Using \Cloud \Shake}\label{sec-using-cloud-shake}

While we expected the GHC build system to be the first to take advantage of
\Cloud \Shake, we were actually beaten to it by Standard Chartered who
reported\footnote{\mbox{\url{https://groups.google.com/d/msg/shake-build-system/NbB5kMFS34I/mZ9L4TgkBwAJ}}}:

\vspace{1mm}
\begin{center}
\parbox{0.8 \textwidth}{\emph{Thanks for the symlinks release, we just finished
upgrading this build system to use \textsf{-}\textsf{-share}. ... Building from
scratch with a warm cache takes around 5 seconds, saving us up to 2 hours. Not
bad!}}
\end{center}
\vspace{1mm}

% We have also got the GHC build working with \Cloud \Shake, resulting in build
% times of 3m and 30s instead of 1h and 8m (on Windows). Due to aspects like slow
% and fully sequential \cmd{configure} we do not see the dramatic reductions
% reported by Standard Chartered.

Converting to a build suitable for sharing is not overly onerous, but nor is it trivial.
In particular, a cloud build is less forgiving about untracked operations -- things that
are wrong but usually harmless in local builds often cause serious problems in a cloud setting.
Some things that require attention in moving to a cloud build:

\begin{itemize}
\item \textbf{Irrelevant differences}: A common problem is that you do not get
      shared caching when you want it. As one example, imagine two users install
      \cmd{gcc} on different paths (say \cmd{/usr/bin/gcc} and \cmd{/usr/local/bin/gcc}).
      If these paths are recorded by the build system, the users won't share
      cache entries. As another example, consider a compiler that embeds the current time
      in the output: any users who build that file locally won't get any shared
      caching of subsequent outputs. Possible solutions include using relative paths;
      depending only on version numbers for system binaries (e.g. \cmd{gcc});
      controlling the environment closely (e.g. using \Nix); and extra flags to
      encourage compilers to be more deterministic.
\item \textbf{Insufficient produced files}: A build rule must declare all files
      it produces, so these can be included in the cache.
      As an example using Haskell, compilation of \cmd{Foo.hs} produces
      \cmd{Foo.hi} and \cmd{Foo.o}. If you declare the rule as producing \cmd{Foo.hi},
      and other rules \emph{depend on} \cmd{Foo.hi}, but \emph{also use} \cmd{Foo.o} after
      depending on \cmd{Foo.hi}, a local build will probably work (although treating \cmd{Foo.o}
      as a proper dependency would definitely be preferable). However,
      if \cmd{Foo.hi} is downloaded from a remote cache, \cmd{Foo.o} will not be present,
      and subsequent commands may fail (e.g. linking).
      In practice, most issues encountered
      during the move to cloud builds for GHC were caused by failing to declare produced
      files.
\item \textbf{Missing dependencies}: While missing dependencies are
      always a problem, the move to a cloud build makes them more serious.
      With local builds, outputs will be built at least once per user, but with a
      cloud build they might only be built \emph{once ever}.
\end{itemize}

To help with the final two issues -- insufficient dependencies and produced
files -- we have further enhanced the \Shake lint modes, coupling them to a
utility called \cmd{FSATrace}, which detects which files are read/written by a
command line execution. Such information has been very helpful in making the GHC
build cloud ready~\cite{cloud_hadrian}.

\subsection{Experience from Building GHC with \Shake}\label{sec-hadrian}

\Hadrian is a build system for the Glasgow Haskell Compiler~\cite{ghc}. It was
developed to replace a \Make-based build system and solve multiple scalability
and maintainability challenges. As discussed in detail
by~Mokhov~\etal~\shortcite{hadrian}, most of these challenges were consequences
of two key shortcomings of \Make: (i)~poor abstraction facilities of makefiles,
notably the need to program in a single namespace of mutable string variables, and
(ii)~the lack of dynamic dependencies~(\S\ref{sec-background-shake}).
\Hadrian benefits both from \Shake's features and from the host language
Haskell, making the new GHC build system easier to understand and maintain.

Interestingly, although \Shake is not a self-tracking build
system~(\S\ref{sec-tracking-aspects}), \Hadrian implements a little
domain-specific language for constructing build command lines, and then tracks
command lines by treating them as a type of values -- an example of
\emph{partial self-tracking} made possible by \Shake's support for key-dependent
value types~(\S\ref{sec-polymorphism}).

The subsequent development of \Cloud \Shake allows GHC developers to take advantage of
sharing build results between builds, which is particularly promising in the
context of GHC's continuous integration (CI) infrastructure responsible for
building and testing every GHC commit. Building GHC~8.8 from scratch with
default settings takes around 1 hour on Windows using either \Hadrian or the
original \Make-based build system. This time includes building the compiler itself,
29~bundled libraries, such as \cmd{base} (each in vanilla and profiled way), and
6~bundled executables, such as \cmd{Haddock}. A single commit typically modifies
only a small part of the whole codebase, making the cloud-build functionality
highly desirable. For example, if a commit does not touch any GHC source files
(e.g. only modifies the documentation), \Hadrian can build GHC from scratch in under
3~minutes, by simply creating symbolic links to the previously built results
stored in the \Cloud~\Shake cache.

A small number of GHC build rules cannot be cached. These rules register
libraries in the package database, and rather than producing files, they
mutate a shared file. \Cloud~\Shake provides a way to manually label such
build rules to exclude them from caching.

One of the benefits of using \Shake is that we have access to high quality build
profiling information, allowing us to compute critical paths and other metrics;
see Mitchell~\shortcite{mitchell2019ghcrebuildtimes} for an overview of \Shake's
profiling features. This information has shown us, for example, that more CPUs would not help (on
unlimited CPUs the speed up would be less than 10\%), and that a handful of
build tasks (two anomalously slow Haskell compilations, and calls to slow
single-threaded \cmd{configure}) take up a significant fraction of build time
(at least 15\%).

% Benchmarking:
% \begin{itemize}
%   \item Building \Hadrian: 55s. For comparison, building the Cabal library: 4m.
%   \item Initial Windows tarball download: 45s.
%   \item Initial \cmd{boot} and \cmd{configure} on Windows: 2m 25s.
%   \item Full GHC build using \Hadrian with default settings:
%         \cmd{hadrian/build.bat -j8}: 57m. Subsequent zero build: 5s.
%   \item Full GHC build using \Hadrian with default settings in the cloud mode:
%         \cmd{hadrian/build.bat -j8 -}\cmd{-share=cache}: 58m.
%         Subsequent zero build: 5s.
%         Subsequent full rebuild after deleting the build directory: 5m 27s.
%   \item Full GHC build using \Make with default settings: \cmd{make -j8}: 60m.
%         Subsequent zero build: 18s.
% \end{itemize}

% The build system used to compile GHC, known as \Hadrian, has continued
% development. The use of dynamic dependencies has made the build much easier, and
% it's now more maintainable. It's now merged into the GHC repo and tested as
% standard by the CI. It's current state is ... In particular, it takes N minutes
% to build with \Shake, vs M minutes to build with \Make. The \Shake version has
% attracted more contributors, and been easier to modify, etc. The code is N
% lines.

% The design we originally outlined remains the one in use. There has been a huge
% amount of engineering work to make the tests pass (many of the details encoded
% in the \Make build system were incidental -- but some were essential).
