\section{Rebuilders}\label{sec-rebuilder}

A build system can be split into a scheduler (as defined in \S\ref{sec-scheduler}) and a \emph{rebuilder}. Suppose the scheduler decides that a key should be brought up to date. The next
question is: does any work need to be done, or is the key already up to date?
Or, in a cloud-build system, do we have a cached copy of the value we need?

While \S\ref{sec-background} explicitly listed the schedulers, the rebuilders were introduced more implicitly, primarily by the information they retain to make their decisions. From the examples we have looked at we see four fundamental rebuilds, each with a number of tweaks and variations within them.

\subsection{A Dirty Bit}\label{sec-dirty-bit}

The idea of a dirty bit is to have one piece of persistent information per key,
saying whether the key is \emph{dirty} or \emph{clean}. After a build, all bits
are set to clean. When the next build starts, anything that changed between the
two builds is marked dirty. If a key and all its transitive dependencies are
clean, the key does not need to be rebuilt. Taking the example from
Figure~\ref{fig-make}, if \hs{util.c} changes then that would be marked dirty, and \hs{util.o} and \hs{main.exe} would rebuild as they transitively depend on \hs{util.c}.

\Excel models the dirty bit approach most directly, having an actual dirty bit
associated with each cell, marking the cell dirty if the user modifies it.
It also marks dirty all cells that (transitively) depend on the modified cell.
\Excel does not record dynamic dependencies of each cell; instead it computes a
\emph{static over-approximation} -- it is safe for it to make more cells dirty
than necessary, but not vice versa. The over-approximation is as follows: a cell
is marked dirty if its formula statically refers to a dirty cell, or if the
formula calls a function like \cmd{INDIRECT} whose dependencies cannot be
guessed from the formula alone. The over-approximation is clear for
\cmd{INDIRECT}, but it is also present for \cmd{IF}, where both branches are
followed even though dynamically only one is used.

\Make uses file modification times, and compares files to their dependencies,
which can be thought of as a dirty bit which is set when a file is older than
its dependencies. The interesting property of this dirty bit is that it is not
under the control of \Make; rather it is existing file-system information that
has been repurposed. Modifying a file automatically clears its dirty bit, and
automatically sets the dirty bit of the files depending on it (but not
recursively). Note that \Make requires that file timestamps only go forward in
time, which can be violated by backup software.

With a dirty bit it is possible to achieve minimality. However, to achieve early
cutoff (\S\ref{sec-background-shake}) it would be important to reset the dirty
bit after a computation that did not change the value and make sure that cells
that depend on it are not rebuilt unnecessarily. For \Excel, this is difficult
because the dependent cells have already been recursively marked dirty. For
\Make, it is impossible to mark a file clean and at the same time not mark the
keys that depend on it dirty. \Make can approximate early cutoff by not
modifying the result file, and not marking it clean, but then it will be rebuilt
in every subsequent build.

The use of a single bit of information is useful to reduce memory consumption, and
in the case of \Make, to integrate with the file system. However, as the examples show,
in constrained environments where a dirty bit is chosen, it's often done as part of
a series of compromises.

\subsection{Verifying Traces}\label{sec-verifying-traces}

An alternative way to determine if a key is dirty is to record the
values/hashes of dependencies used last time, and if something has changed, the
key is dirty and must be rebuilt~--~in essence a \emph{trace} which we can use
to \emph{verify} existing values. Taking the example from Figure \ref{fig-make},
we might record that \hs{util.o} dependend on \hs{util.c} (at hash 1) and \hs{util.h}
(at hash 2) and produced \hs{util.o} (at hash 4). Next time round, if all those files
still have the same hashes, there is nothing to do, and we can skip running the build.
If they are different, we repeat the action, and record the new values.

For traces, there are two essential
operations~--~adding a new value to the trace store, and using the traces to
determine if a key needs rebuilding. Assuming a store of verifying traces
\hs{VT}~\hs{k}~\hs{v}, the operations are:

\begin{minted}[fontsize=\small,xleftmargin=10pt]{haskell}
recordVT@\,@::@\,@k -> Hash v -> [(k, Hash v)] -> VT k v -> VT k v
verifyVT@\,@::@\,@(Monad m,@\,@Eq k,@\,@Eq v) => k -> Hash v -> (k -> m (Hash v)) -> VT k v -> m Bool
\end{minted}

\noindent
Rather than storing (large) values \hs{v}, the verifying trace \hs{VT} can store
only hashes, of type \hs{Hash}~\hs{v}, of those values. Since the verifying
trace persists from one build to the next -- it constitutes the build system's
``memory'' -- it is helpful for it to be of modest size. After successfully
building a key, we call \hs{recordVT} to add a record to the current \hs{VT},
passing the key, the hash of its value, and the list of hashes and dependencies.

More interestingly, to \emph{verify} whether a key needs rebuilding we supply
the key, the hash of its current value, a function for obtaining the post-build
value of any key (using a scheduling strategy as per
\S\ref{sec-dependency-orderings}), and the existing \hs{VT} information. The
result will be a \hs{Bool} where \hs{True} indicates that the current value is
already up to date, and \hs{False} indicates that it should be rebuilt.

A verifying trace, and other types of traces discussed in this section, support
dynamic dependencies and minimality; furthermore, all traces except for deep
traces~(\S\ref{sec-deep-constructive-traces}) support early cutoff.

\subsubsection{Representation Optimisations}

One potential implementation would be to record all arguments passed to
\hs{recordVT} in a list, and verify by simply checking if any list item matches
the information passed by \hs{verifyVT}. Concretely,
in our implementations from \S\ref{sec-implementations}, traces are recorded as lists of:

\begin{minted}[xleftmargin=10pt]{haskell}
data Trace k v = Trace { key :: k, depends :: [(k, Hash v)], result :: Hash v }
\end{minted}

\noindent Where \hs{r} is \hs{Hash}~\hs{v} for verifying traces (and \hs{v} for constructive traces, discussed later in \S\ref{sec-constructive-traces}). A real system is highly likely to use a more optimised implementation.

The first optimisation is that any system using \hs{Applicative} dependencies can omit the dependency
keys from the \hs{Trace} since they can be recovered from the \hs{key} field.

The next optimisation is that there is only very minor benefit from storing more than one \hs{Trace} per key. Therefore, veryifying traces can be stored as \hs{Map}~\hs{k}~\hs{(Trace}~\hs{k}~\hs{v}~\hs{(Hash}~\hs{v))}, where the
initial \hs{k} is the \hs{key} field of \hs{Trace} -- making \hs{verifyVT} fast.
Note that storing only one \hs{Trace} per key means that if the dependencies change but the result does not, and then the
dependencies change back to what they were before, there will not be a valid \hs{Trace} available and the system will have to rebuild, where a complete list of all historical traces would not have had to. However, it also means that the number of \hs{Trace} structures is bounded by the number of distinct keys, regardless of how many builds are executed -- a useful property.

\subsubsection{Verifying Step Traces}\label{sec-step-traces}

The \Shake build system, and the associated paper \cite{mitchell2012shake} in \S2.3.3 describes a slightly different trace structure, one that is similar to verifying traces, but stores slightly less data, and has slightly different early cutoff semantics. Rather than storing the \hs{Hash v} for each dependency, it instead stores a built time and changed time for each key, and for the dependencies just stores a list of \hs{k}. The resulting \hs{Trace} type ressembles:

\begin{minted}[fontsize=\small,xleftmargin=10pt]{haskell}
data Trace k v = Trace
    {key :: k
    ,result :: Hash v
    ,depends :: [k]
    ,built :: Time
    ,changed :: Time
    }
\end{minted}

The \hs{built} field is when the \hs{key} last rebuilt. The \hs{changed} field is when the \hs{result} last changed -- if the last build changed the value, it will be equal to \hs{built}, otherwise it will be older. The \hs{recordVT} function needs to consult the previous \hs{Trace} to know whether to use \hs{built} or the previous \hs{changed} value as the new \hs{changed} value. The \hs{verifyVT} needs to verify that the \hs{built} time of \hs{key} is newer than the \hs{changed} values of all its \hs{depends}. This scheme preserves minimality and early cutoff. A variant with only one \hs{Time} field would loose early cutoff, and indeed corresponds quite closely to \Make. Furthermore, the \hs{Time} stamp only needs record which execution of the build is running, so every key built in the same run can share the same \hs{Time} value -- it just needs to be monotically increasing between runs.

This optimisation is useful, at least in the case of \Shake, to save space. A hash takes up 32 bytes in typical programs, while a key (in \Shake only) is an \hs{Int} taking only 4 bytes. Furthermore, \Shake permits values to be arbitrarily large, and permits a custom equality (two values can be bit-for-bit unequal but considered equal by \Shake), meaning that a \hs{Hash} isn't a valid encoding. For build systems that used \hs{Applicative} dependencies, the \hs{depends} could be omitted entirely, making the size of a \hs{Trace} $O(1)$ instead of $O(n)$ where $n$ is the number of dependencies.

While verifying step traces are mostly an optimisation, there are some observable differences from verifying traces. Considering Figure \ref{fig-make}, imagine we built \hs{main.exe} (everything gets a \hs{built} and \hs{changed} of timestamp 1). Next we change \hs{util.c} and rebuild \hs{util.o} (both get a \hs{built} and \hs{changed} of timestamp 2). Next we put \hs{util.c} back to what it was originally, and build \hs{main.exe}. With verifying traces, the dependencies of \hs{main.exe} would be equal, and \hs{main.exe} would not rebuild. With verifying step traces, the \hs{changed} field of \hs{util.o} would increase once more, and \hs{main.exe} would rebuild. Other than when building subsets of the targets, we are unaware of any other situation where verifying step traces are less powerful.

\subsection{Constructive Traces}\label{sec-constructive-traces}

A verifying trace deliberately records only small hashes, so that it can be small.
In contrast, a \emph{constructive} trace also stores the resulting value. Concretely,
it records a list of \hs{Trace}~\hs{k}~\hs{v}.
Once we are storing the complete result it makes sense
to record many constructive traces per key, and to share them with other users,
providing cloud-build functionality. We can represent this additional
information with the operations:

\begin{minted}[fontsize=\small,xleftmargin=10pt]{haskell}
recordCT    :: k -> v -> [(k, Hash v)] -> CT k v -> CT k v
constructCT :: (Monad m, Eq k, Eq v) => k -> (k -> m (Hash v)) -> CT k v -> m [v]
\end{minted}

\noindent
The function \hs{recordCT} looks like \hs{recordVT}, but instead of just passing
the hash of the resulting value, we require the actual value. The \hs{verifyVT}
has been replaced with \hs{constructCT}, which instead of taking the hash of the
current value as \emph{input}, returns a list of suitable values as \emph{output}.
If the current value in the store matches one of the possible values, the build
can skip this key. If the resulting list is empty, the key must be rebuilt.
However, if the current value does not match the store, and there is a possible
value, we can use any value from the constructive list \emph{without} doing any
work to build it, and copy it into the store.

Any \hs{Applicative} build system using constructive traces, e.g.
\CloudBuild, can index directly from the key and results to the output result~--~i.e.
\hs{Map}~\hs{(@@k,}~\hs{[Hash}~\hs{v])}~\hs{v}. Importantly, assuming the traces
are stored on a central server, the client can compute the key and the hashes of
its dependencies, then make a single call to the server to retrieve the result.

In practice, many cloud build systems store hashes of values in the trace information,
then have a separate content-addressable cache which associates hashes with
their actual contents.

\subsection{Deep Constructive Traces}\label{sec-deep-constructive-traces}

Constructive traces always verify keys by looking at their immediate
dependencies, which must have first been brought up to date, meaning that the
time to verify a key depends on the number of transitive dependencies. A
\emph{deep} constructive trace optimises this process by only looking at the
terminal \emph{input keys}, ignoring any intermediate dependencies. The operations
capturing this approach are the same as for constructive traces
in~\S\ref{sec-constructive-traces}, but we use the names \hs{recordDCT} and
\hs{constructDCT}, where the underlying \hs{DCT} representation need only record
information about hashes of inputs, not intermediate dependencies.

Concretely, taking the example from Figure \ref{fig-make}, to decide whether \hs{main.exe}
is out of data, a \emph{constructive} trace would look at \hs{util.o} and \hs{main.o}
(the immediate dependencies), while a \hs{deep constructive} trace would look at \hs{util.c},
\hs{util.h} and \hs{main.c}.

An advantage of deep constructive traces is that to decide if \hs{main.exe} is up to date only
requires consulting its inputs, not even considering \hs{util.o} or \hs{main.o}.
Such a feature is often known as a \emph{shallow build}, as discussed in~\S\ref{sec-cloud-aspects}.

There are two primary disadvantages of deep constructive traces:

\begin{description}
\item[Deterministic tasks] If the tasks are not \emph{deterministic} then it is possible to
violate correctness, as illustrated by an example in~\S\ref{sec-cloud-aspects}
(see Fig.~\ref{fig-frankenbuild}).
\item[No early cutoff] Such traces cannot support early cutoff
(\S\ref{sec-background-shake}), since the results of intermediate computations are not considered.
\end{description}

Current build systems using deep constructive traces always record hashes of
terminal \emph{input keys}, but the technique works equally well if we skip any
number of dependency levels (say $n$ levels). The input-only approach is the
special case of $n = \infty$, and constructive traces are the special case of
$n = 1$. By picking values of $n$ in between we would regain some early cutoff, at the
cost of losing such simple shallow builds and still requiring determinstic tasks.
