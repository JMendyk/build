\section{Schedulers}\label{sec-scheduler}

The focus of this paper is on a variety of implementations of
\hs{Build}~\hs{c}~\hs{i}~\hs{k}~\hs{v}, given a \emph{user-supplied}
implementation of \hs{Tasks}~\hs{c}~\hs{k}~\hs{v}. That is, we are going to take
\hs{Tasks} as given from now on, and explore variants of \hs{Build}: first
abstractly (in this section and in~\S\ref{sec-rebuilder}) and then concretely
in~\S\ref{sec-implementations}.

As per the definition of minimality~(\S\ref{def-minimal}), a minimal build
system must \textbf{rebuild only out-of-date keys} and at most once. The only
way to achieve the ``at most once'' requirement while producing a correct build
result (\S\ref{sec-build-correctness}) is to \textbf{build all keys in an
order that respects their dependencies}.

We have emboldened two different aspects above: the part of the build system
responsible for scheduling tasks in the dependency order (a ``scheduler'') can
be cleanly separated from the part responsible for deciding whether a key needs
to be rebuilt (a ``rebuilder''). In this section we discuss schedulers, leaving
rebuilders for \S\ref{sec-rebuilder}.

Section \S\ref{sec-background} introduced three different \emph{task schedulers}
that decide which tasks to execute and in what order; see the ``Scheduler''
column of Table~\ref{tab-summary} in \S\ref{sec-background-summary}. The
following subsections explore the properties of the three schedulers, and
possible implementations.

\subsection{Topological Scheduler}\label{sec-topological}

The topological scheduler pre-computes a linear order of tasks, which when
followed ensures dependencies are satisfied, then executes the required tasks in
that order. Computing such a linear order is straightforward -- given a task
description and a target \hs{key}, first find the (acyclic) graph of the
\hs{key}'s dependencies, then compute a topological order. Taking the \Make
example from Fig.~\ref{fig-make}, we might compute the following order:

\begin{enumerate}
\item \cmd{main.o}
\item \cmd{util.o}
\item \cmd{main.exe}
\end{enumerate}

\noindent
Given the dependencies, we could have equally chosen to build \cmd{util.o}
first, but \cmd{main.exe} \emph{must} come last.

The advantage of this scheme is simplicity -- compute an order, then execute
tasks in that order. In addition, any missing keys or dependency cycles can be
detected from the graph, and reported to the user before any work has commenced.

The downside of this approach is that it requires the dependencies of each task
in advance. As we saw in~\S\ref{sec-deps}, we can only extract dependencies from
an applicative task, which requires the build system to choose
\hs{c}~\hs{=}~\hs{Applicative}, ruling out dynamic dependencies.

\subsection{Restarting Scheduler}\label{sec-restarting}

To handle dynamic dependencies we cannot precompute a static order -- we must
interleave running tasks and ordering tasks. One approach is just to build tasks
in an arbitrary order, and if a task calls \hs{fetch} on an out-of-date key
\cmd{dep}, abort the task and build \cmd{dep} instead. Returning to the example
from Fig.~\ref{fig-make}, we might build the tasks as follows:

\begin{enumerate}
\item \cmd{main.exe} (abort because it depends on \cmd{util.o} which is out of
      date)
\item \cmd{main.o}
\item \cmd{util.o}
\item \cmd{main.exe} (restart from scratch, completing successfully this time)
\end{enumerate}

\noindent
We start with \cmd{main.exe} (an arbitrary choice), but discover it depends on
\cmd{main.o}, so instead start building \cmd{main.o}. Next we choose to build
\cmd{util.o} (again, arbitrarily), before finally returning to \cmd{main.exe}
that now has all its dependencies available and completes successfully.

This approach works, but has a number of disadvantages. Firstly, it requires a
technical mechanism to abort a task, which is easy in our theoretical setting
with \hs{Task} (see an implementation in~\S\ref{sec-implementation-excel}) but
leads to engineering concerns in the real world. Secondly, it is not minimal in
the sense that a task may start, do some meaningful work, and then abort,
repeating that same work when restarted.

As a refinement, to reduce the number of aborts (often to zero) \Excel records
the discovered task order in its \emph{calc chain}, and uses it as the starting
point for the next build~(\S\ref{sec-background-excel}). \Bazel's restarting
scheduler does not store the discovered order between build runs; instead, it
stores the most recent task dependency information from which it can compute a
linear order. Since this information may become outdated, \Bazel may also need
to abort a task if a newly discovered dependency is out of date.

\subsection{Suspending Scheduler}\label{sec-suspending}

An alternative approach, utilised by the \hs{busy} build system
(\S\ref{sec-general-build}) and \Shake, is to simply build dependencies when
they are requested, suspending the currently running task when needed. Using the
example from Fig.~\ref{fig-make}, we would build:

\begin{itemize}
\item \cmd{main.exe} (suspended)\\
  $\hookrightarrow$ \cmd{main.o}
\item \cmd{main.exe} (resumed then suspended again)\\
  $\hookrightarrow$   \cmd{util.o}
\item \cmd{main.exe} (completed)
\end{itemize}

\noindent
We start building \cmd{main.exe} first as it is the required target. We soon
discover a dependency on \cmd{main.o} and suspend the current task
\cmd{main.exe} to build \cmd{main.o}, then resume and suspend again to build
\cmd{util.o}, and finally complete the target \cmd{main.exe}.

This scheduler (when combined with a suitable rebuilder) provides a minimal
build system that supports dynamic dependencies. In our model, a suspending
scheduler is easy to write -- it makes a function call to compute each
dependency. However, a more practical implementation is likely to build multiple
dependencies in parallel, which then requires a more explicit task suspension
and resumption. To implement suspension there are two standard approaches:

\begin{itemize}
\item Blocking threads or processes. This approach is relatively easy, but can
require significant resources, especially if a large number of tasks are
suspended. In languages with cheap green threads (e.g. Haskell) the approach is
more feasible, and it was the original approach taken by \Shake.
\item Continuation-passing style~\cite{claessen_continuations} can allow the
remainder of a task to be captured, paused, and resumed later. Continuation
passing is efficient, but requires the build script to be architected to allow
capturing continuations. \Shake currently uses this approach.
\end{itemize}

\noindent
While a suspending scheduler is theoretically optimal, in practice it is better
than a restarting scheduler only if the cost of avoided duplicate work
outweighs the cost of suspending tasks. Note furthermore that the cost of
duplicate work may often be just a fraction of the overall build cost.
