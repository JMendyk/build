\section{Introduction}\label{sec-intro}

Build systems (such as \Make) are big, complicated, and used by every
software developer on the planet.  But they are a sadly unloved part
of the software ecosystem, very much a means to an end, and seldom the
focus of attention.
% Rarely do people ask questions like ``What does it mean for my build
% system to be correct?'' or ``What are the trade-offs between different
% approaches?''.
For years \Make dominated, but more recently the challenges of scale have driven
large software firms like Microsoft, Facebook and Google to develop their own
build systems, exploring new points in the design space. These complex build
systems use subtle algorithms, but they are often hidden away, and not the
object of study.

In this paper we give a general framework in which to understand and compare
build systems, in a way that is both abstract (omitting incidental detail) and
yet precise (implemented as Haskell code). Specifically we make these
contributions:

\begin{itemize}
\item Build systems vary on many axes, including: static vs dynamic
  dependencies; local vs cloud; deterministic vs non-deterministic build tasks;
  early cutoff; self-tracking build systems; and the type of persistently stored
  build information. In~\S\ref{sec-background} we identify some of these key
  properties, illustrated by four carefully-chosen build systems.

\item We describe some simple but novel abstractions that
  crisply encapsulate what a build system is (\S\ref{sec-abstractions}),
  allowing us, for example, to speak about what it means for a build system to
  be correct.

\item We identify two key design choices that are typically deeply wired into
  any build system: \emph{the order in which tasks are
  built}~(\S\ref{sec-scheduler}) and \emph{whether or not a
  task is rebuilt}~(\S\ref{sec-rebuilder}). These choices turn out to
  be orthogonal, which leads us to a new classification of the design
  space~(\S\ref{sec-design-space}).

\item We show that we can instantiate our abstractions to describe the essence
of a variety of different real-life build systems, including \Make, \Shake,
\Bazel, \Buck, \Nix, and \Excel\footnote{\Excel appears very different to the
others but, seen through the lens of this paper, it is very close.}, each by the
composition of the two design choices~(\S\ref{sec-implementations}). Doing this
modelling in a single setting allows the differences and similarities between
these huge systems to be brought out clearly\footnote{All our models are
executable and are available on Hackage as \cmd{build-1.0}.}.

\item Moreover, we can readily remix the ingredients to design new build systems
with desired properties, for example, to combine the advantages of \Shake and
\Bazel. Writing this paper gave us the insights to combine dynamic dependencies and
cloud build in a principled way; we evaluate the result~(\S\ref{sec-experience}).

\item We can use these new abstractions to more clearly explain details from the
original \Shake paper~\cite{mitchell2012shake} (\S\ref{sec-step-traces},
\S\ref{sec-experience-shake}) and build new features useful for the GHC build
system~\cite{hadrian} (\S\ref{sec-using-cloud-shake}).

\end{itemize}

In short, instead of seeing build systems as unrelated points in space, we now
see them as locations in a connected landscape, leading to a better
understanding of what they do and how they compare, and making it easier to
explore other points in the landscape. While we steer clear of many engineering
aspects of real build systems, in~\S\ref{sec-engineering} we show how they can
be incorporated into the presented abstractions. The related work is discussed
in~\S\ref{sec-related}.

This paper is an extended version of an earlier conference
paper~\cite{mokhov2018buildsystems}. The key changes compared to the earlier
version are: (i) we added further clarifications and examples
into~\S\ref{sec-abstractions}, in particular,~\S\ref{sec-task-examples} is
entirely new; (ii)~\S\ref{sec-scheduler} and~\S\ref{sec-rebuilder} are based on
the material from the conference paper but have been substantially expanded to
include further details and examples, as well as completely new material such
as~\S\ref{sec-step-traces}; (iii)~\S\ref{sec-experience} is completely new; (iv)
\S\ref{sec-failures} and \S\S\ref{sec-polymorphism}-\ref{sec-tracking-aspects}
are almost entirely new, and \S\ref{sec-non-determinism} has been revised.
The new material focuses on our experience and various important practical
considerations, hence justifying the ``and Practice'' part of the paper title.
